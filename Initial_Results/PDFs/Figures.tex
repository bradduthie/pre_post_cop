\documentclass[12pt]{article}
\usepackage[top=1.25in, bottom=1in, left=1.25in, right=1in]{geometry}
\usepackage{amssymb}
\usepackage{amsmath}
\usepackage{setspace}
\usepackage{natbib}
\usepackage{rotating}
\usepackage{graphicx}
\usepackage{multirow}
\usepackage{lineno}
\usepackage{datetime}
\setkomafont{\rmfamily\bfseries\boldmath}
\usepackage{wrapfig,floatrow}
\usepackage{float}
\usepackage{fancyhdr}
\usepackage[font=small,labelfont=bf]{caption}
\usepackage{mathabx}
\usepackage{color}
\usepackage{wasysym}
\usepackage{lipsum}
\floatstyle{plain}
\restylefloat{figure}

\newcommand*{\TitleFont}{
      \usefont{\encodingdefault}{\rmdefault}{r}{n}
      \fontsize{16}{20}
      \selectfont}

\usepackage{fancyheadings}
\pagestyle{fancyplain}
\fancyhf{} 
\renewcommand{\headrulewidth}{0pt} 
\fancyfoot[RO]{\thepage}

\makeatletter
\renewcommand\section{\@startsection{section}{1}{0in}{-0.5\baselineskip}{0.1\baselineskip}{\normalfont\large\bfseries}}
\makeatother

\makeatletter
\renewcommand\subsection{\@startsection{subsection}{1}{-0.25in}{-0.5\baselineskip}{0.1\baselineskip}{\normalfont\normalsize\bfseries\textit}}
\makeatother

\makeatletter
\renewcommand\subsubsection{\@startsection{subsubsection}{1}{-0.25in}{-0.5\baselineskip}{0.1\baselineskip}{\normalfont\normalsize\textit}}
\makeatother

\title{Figures for pre and postcopulatory inbreeding behaviour}
\author{A. Bradley Duthie\textsuperscript{1,*}, Other Authors\textsuperscript{1}} 

\date{}


\pagestyle{fancy}
\renewcommand{\headrulewidth}{0pt}

\begin{document}
\maketitle

\begin{center}
\vspace{5 mm}

\noindent 1. Institute of Biological and Environmental Sciences, School of Biological Sciences, Zoology Building, Tillydrone Avenue, University of Aberdeen, Aberdeen AB24 2TZ, United Kingdom \textsuperscript{*} E-mail: aduthie@abdn.ac.uk \\ 
\newline
\end{center}

\vspace{15 mm}

\noindent\textbf{The following figures present results for some initial modelling of evolving traits underlying 1) pre-copulatory inbreeding preference or avoidance, 2) tendency for engaging in polyandry, and 3) post-copulatory inbreeding preference or avoidance (note that post-copulatory inbreeding preference or avoidance is contingent upon focal females being polyandrous. Evolutionary outcomes in Figures 1-16 show evolutionary consequences -- specifically with respect to the mean values of alleles underlying traits -- for inbreeding behaviour and polyandry after 5000 generations. Figures 17-20 show changing mean allele values over generations. Positive allele values underly either inbreeding preference or positive tendency to engage in polyandry, and negative allele values underly either inbreeding avoidance or monandry.}

%\linenumbers
%\modulolinenumbers[2]
%\doublespacing

% -----------------------------------------------------------------------------------

\begin{figure}
\begin{center}				
\includegraphics[scale=0.5]{EC_Beta_0.pdf}
\end{center}
\caption{Four evolutionary outcomes given three randomised costs. \textbf{Initial trait values are set to zero.}}
\end{figure}

\begin{figure}
\begin{center}				
\includegraphics[scale=0.5]{PO_Beta_0.pdf}
\end{center}
\caption{Evolution of polyandry versus monandry given three randomised costs. \textbf{Initial trait values are set to zero.}}
\end{figure}

\clearpage

% -----------------------------------------------------------------------------------

\begin{figure}
\begin{center}				
\includegraphics[scale=0.5]{EC_Beta_1.pdf}
\end{center}
\caption{Four evolutionary outcomes given three randomised costs. \textbf{Initial trait values are set to zero.}}
\end{figure}

\begin{figure}
\begin{center}				
\includegraphics[scale=0.5]{PO_Beta_1.pdf}
\end{center}
\caption{Evolution of polyandry versus monandry given three randomised costs. \textbf{Initial trait values are set to zero.}}
\end{figure}

\clearpage

% -----------------------------------------------------------------------------------

\begin{figure}
\begin{center}				
\includegraphics[scale=0.5]{EC_Beta_2.pdf}
\end{center}
\caption{Four evolutionary outcomes given three randomised costs. \textbf{Initial trait values are set to zero.}}
\end{figure}

\begin{figure}
\begin{center}				
\includegraphics[scale=0.5]{PO_Beta_2.pdf}
\end{center}
\caption{Evolution of polyandry versus monandry given three randomised costs. \textbf{Initial trait values are set to zero.}}
\end{figure}

\clearpage

% -----------------------------------------------------------------------------------

\begin{figure}
\begin{center}				
\includegraphics[scale=0.5]{EC_Beta_3.pdf}
\end{center}
\caption{Four evolutionary outcomes given three randomised costs. \textbf{Initial trait values are set to zero.}}
\end{figure}

\begin{figure}
\begin{center}				
\includegraphics[scale=0.5]{PO_Beta_3.pdf}
\end{center}
\caption{Evolution of polyandry versus monandry given three randomised costs. \textbf{Initial trait values are set to zero.}}
\end{figure}

\clearpage

% -----------------------------------------------------------------------------------

\begin{figure}
\begin{center}				
\includegraphics[scale=0.5]{EC_Beta_0_z1.pdf}
\end{center}
\caption{Four evolutionary outcomes given three randomised costs. \textbf{Initial trait values are randomised $\mathcal{N}\left(0,1\right)$.}}
\end{figure}

\begin{figure}
\begin{center}				
\includegraphics[scale=0.5]{PO_Beta_0_z1.pdf}
\end{center}
\caption{Evolution of polyandry versus monandry given three randomised costs. \textbf{Initial trait values are randomised $\mathcal{N}\left(0,1\right)$.}}
\end{figure}

\clearpage

% -----------------------------------------------------------------------------------

\begin{figure}
\begin{center}				
\includegraphics[scale=0.5]{EC_Beta_1_z1.pdf}
\end{center}
\caption{Four evolutionary outcomes given three randomised costs. \textbf{Initial trait values are randomised $\mathcal{N}\left(0,1\right)$.}}
\end{figure}

\begin{figure}
\begin{center}				
\includegraphics[scale=0.5]{PO_Beta_1_z1.pdf}
\end{center}
\caption{Evolution of polyandry versus monandry given three randomised costs. \textbf{Initial trait values are randomised $\mathcal{N}\left(0,1\right)$.}}
\end{figure}

\clearpage

% -----------------------------------------------------------------------------------

\begin{figure}
\begin{center}				
\includegraphics[scale=0.5]{EC_Beta_2_z1.pdf}
\end{center}
\caption{Four evolutionary outcomes given three randomised costs. \textbf{Initial trait values are randomised $\mathcal{N}\left(0,1\right)$.}}
\end{figure}

\begin{figure}
\begin{center}				
\includegraphics[scale=0.5]{PO_Beta_2_z1.pdf}
\end{center}
\caption{Evolution of polyandry versus monandry given three randomised costs. \textbf{Initial trait values are randomised $\mathcal{N}\left(0,1\right)$.}}
\end{figure}

\clearpage

% -----------------------------------------------------------------------------------

\begin{figure}
\begin{center}				
\includegraphics[scale=0.5]{EC_Beta_3_z1.pdf}
\end{center}
\caption{Four evolutionary outcomes given three randomised costs. \textbf{Initial trait values are randomised $\mathcal{N}\left(0,1\right)$.}}
\end{figure}

\begin{figure}
\begin{center}				
\includegraphics[scale=0.5]{PO_Beta_3_z1.pdf}
\end{center}
\caption{Evolution of polyandry versus monandry given three randomised costs. \textbf{Initial trait values are randomised $\mathcal{N}\left(0,1\right)$.}}
\end{figure}

\clearpage

% -----------------------------------------------------------------------------------

\begin{figure}
\begin{center}				
\includegraphics[scale=0.5]{evo_no_cost_B0.pdf}
\end{center}
\caption{Evolution of mean trait values over time. \textbf{Initial trait values are set to zero.}}
\end{figure}

\begin{figure}
\begin{center}				
\includegraphics[scale=0.5]{evo_no_cost_B1.pdf}
\end{center}
\caption{Evolution of mean trait values over time. \textbf{Initial trait values are set to zero.}}
\end{figure}

\clearpage

% -----------------------------------------------------------------------------------

\begin{figure}
\begin{center}		
\includegraphics[scale=0.5]{evo_no_cost_B2.pdf}
\end{center}
\caption{Evolution of mean trait values over time. \textbf{Initial trait values are set to zero.}}
\end{figure}

\begin{figure}
\begin{center}				
\includegraphics[scale=0.5]{evo_no_cost_B3.pdf}
\end{center}
\caption{Evolution of mean trait values over time. \textbf{Initial trait values are set to zero.}}
\end{figure}

\clearpage

% -----------------------------------------------------------------------------------

\begin{figure}
\begin{center}				
\includegraphics[scale=0.5]{evo_no_cost_B0_z1.pdf}
\end{center}
\caption{Evolution of mean trait values over time. \textbf{Initial trait values are randomised $\mathcal{N}\left(0,1\right)$.}}
\end{figure}

\begin{figure}
\begin{center}				
\includegraphics[scale=0.5]{evo_no_cost_B1_z1.pdf}
\end{center}
\caption{Evolution of mean trait values over time. \textbf{Initial trait values are randomised $\mathcal{N}\left(0,1\right)$.}}
\end{figure}

\clearpage

% -----------------------------------------------------------------------------------

\begin{figure}
\begin{center}		
\includegraphics[scale=0.5]{evo_no_cost_B2_z1.pdf}
\end{center}
\caption{Evolution of mean trait values over time. \textbf{Initial trait values are randomised $\mathcal{N}\left(0,1\right)$.}}
\end{figure}

\begin{figure}
\begin{center}				
\includegraphics[scale=0.5]{evo_no_cost_B3_z1.pdf}
\end{center}
\caption{Evolution of mean trait values over time. \textbf{Initial trait values are randomised $\mathcal{N}\left(0,1\right)$.}}
\end{figure}

\clearpage

% -----------------------------------------------------------------------------------

\end{document}














