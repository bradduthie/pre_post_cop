\documentclass[10pt,letterpaper]{article}
\usepackage[top=0.85in,left=2.75in,footskip=0.75in,marginparwidth=2in]{geometry}
\usepackage{amssymb}
\usepackage{amsmath}
\usepackage{setspace}
\usepackage{natbib}
\usepackage{rotating}
\usepackage{multirow}
\usepackage{datetime}
\setkomafont{\rmfamily\bfseries\boldmath}
\usepackage{wrapfig,floatrow}
\usepackage{float}
\usepackage[font=small,labelfont=bf]{caption}
\usepackage{mathabx}
\usepackage{color}
\usepackage{wasysym}
\usepackage{sidecap}
%\usepackage{caption}
\usepackage{xargs,caption,changepage,ifthen}
\usepackage[demo]{graphicx}

% use Unicode characters - try changing the option if you run into troubles with special characters (e.g. umlauts)
\usepackage[utf8]{inputenc}

% clean citations
%\usepackage{cite}

% hyperref makes references clicky. use \url{www.example.com} or \href{www.example.com}{description} to add a clicky url
\usepackage{nameref,hyperref}
\hypersetup{
     colorlinks   = true,
     citecolor    = blue,
     linkcolor    = black
}

% line numbers
\usepackage[right]{lineno}

% improves typesetting in LaTeX
\usepackage{microtype}
\DisableLigatures[f]{encoding = *, family = * }

% text layout - change as needed
%\raggedright
\setlength{\parindent}{0.5cm}
\textwidth 5.25in 
\textheight 8.75in

% Remove % for double line spacing
%\usepackage{setspace} 
%\doublespacing

% use adjustwidth environment to exceed text width (see examples in text)
\usepackage{changepage}

% adjust caption style
\usepackage[aboveskip=1pt,labelfont=bf,labelsep=period,singlelinecheck=off]{caption}

% remove brackets from references
\makeatletter
\renewcommand{\@biblabel}[1]{\quad#1.}
\makeatother

% headrule, footrule and page numbers
\usepackage{lastpage,fancyhdr,graphicx}
\usepackage{epstopdf}
\pagestyle{myheadings}
\pagestyle{fancy}
\fancyhf{}
\rfoot{\thepage/\pageref{LastPage}}
\renewcommand{\footrule}{\hrule height 2pt \vspace{2mm}}
\fancyheadoffset[L]{2.25in}
\fancyfootoffset[L]{2.25in}

% use \textcolor{color}{text} for colored text (e.g. highlight to-do areas)
\usepackage{color}

% define custom colors (this one is for figure captions)
\definecolor{Gray}{gray}{.25}

% this is required to include graphics
\usepackage{graphicx}

% use if you want to put caption to the side of the figure - see example in text
\usepackage{sidecap}

% use for have text wrap around figures
\usepackage{wrapfig}
\usepackage[pscoord]{eso-pic}
\usepackage[fulladjust]{marginnote}
\reversemarginpar



%----------------------------------------
%COMMAND FOR DOING SIDE CAPTION.
%----------------------------------------
\newcommandx{\mycaptionminipage}[3][3=c,usedefault]{%
    \begin{minipage}[#3]{#1}%
        \ifthenelse{\equal{#3}{b}}{\captionsetup{aboveskip=0pt}}{}
        \ifthenelse{\equal{#3}{t}}{\captionsetup{belowskip=0pt}}{}
        \vspace{0pt}\centering\captionsetup{width=\textwidth} %Temporarily set caption width
        #2%
    \end{minipage}%
}%
\newcommandx{\mysidecaption}[4][4=c,usedefault]{%
    \checkoddpage%
    \ifoddpage%
        %CASE ODD PAGES
        \mycaptionminipage{\dimexpr\linewidth-#1\linewidth-\intextsep\relax}{#3}[#4]%
        \hfill%
        \mycaptionminipage{#1\linewidth}{#2}[#4]%
    \else%
        %CASE EVEN PAGES
        \mycaptionminipage{#1\linewidth}{#2}[#4]%
        \hfill%
        \mycaptionminipage{\dimexpr\linewidth-#1\linewidth-\intextsep\relax}{#3}[#4]%
    \fi%
}%


% document begins here
\begin{document}
\vspace*{0.35in}

% title goes here:
\begin{flushleft}
{\LARGE
\textbf\newline{Evolution of pre-copulatory and post-copulatory inbreeding behaviour}
}
\newline
% authors go here:
\\
A. Bradley Duthie\textsuperscript{1,*},
Jane M. Reid\textsuperscript{1}
\\
\bigskip
\bf{1} Institute of Biological and Environmental Sciences, School of Biological Sciences, Zoology Building, Tillydrone Avenue, University of Aberdeen, Aberdeen AB24 2TZ, United Kingdom
%\\
%\bf{2} Affiliation B
\\
\bigskip
*  aduthie@abdn.ac.uk, brad.duthie@gmail.com

\end{flushleft}

\section*{Abstract}
\marginpar{
\vspace{0.0cm} % adjust vertical position relative to text with \vspace{} - note that you can enter negative numbers to move margin caption up
\color{Gray} % this gives caption a grey color to set it apart from text body
\textbf{Key words:} % note that \ref{fig1} refers to the corresponding wrapfigure
Inbreeding, inbreeding avoidance, inbreeding depression, mate choice, relatedness, reproductive strategy
}
Inbreeding avoidance has been widely observed through pre-copulatory and post-copulatory mechanisms as a consequence of inbreeding depression in offspring viability. However, no conceptual framework compares evolution of pre-copulatory and post-copulatory inbreeding avoidance jointly, and given biologically relevant costs of inbreeding strategies. Critically, evolution of post-copulatory inbreeding avoidance is restricted to females that mate multiply, potentially leading to complex evolutionary dynamics among inbreeding strategies and polyandry. Accordingly, we use individual-based modelling to track evolution of polyandry and inbreeding strategies under multiple costs and constraints to address three questions: Will evolution of post-copulatory inbreeding avoidance strengthen selection for polyandry? Will selection for a costly inbreeding strategy be negligible given a non-costly alternative strategy? And will selection for an inbreeding strategy be negligible given existence of an already adaptive alternative strategy? We show that selection for polyandry is greatly strengthened by post-copulatory inbreeding avoidance. Strong selection for relatively low cost inbreeding avoidance strategies was observed, while selection for functionally redundant high cost strategies decreased over generations. Finally, existence of an adaptive inbreeding strategy precluded evolution of post-copulatory, but not pre-copulatory, inbreeding avoidance. Our model thereby introduces a novel framework for predicting evolution of both pre-copulatory and post-copulatory inbreeding strategies in empirical systems.

\section*{Introduction}

Inbreeding typically reduces the fitness of inbred offspring (termed `inbreeding depression'), potentially driving evolution of inbreeding avoidance when inbreeding depression is strong \cite[][]{Parker1979, Parker2006, Pusey1996, Charlesworth1999, Keller2002, Charlesworth2009, Szulkin2012}. For focal females, inbreeding avoidance might be acheived by avoiding related potential mates (pre-copulatory inbreeding avoidance), or by biasing fertilisation after mating has occurred so that related mates are less likely to sire offspring (post-copulatory inbreeding avoidance). Importantly, evolution of post-copulatory inbreeding avoidance requires that females be polyandrous, mating with multiple males during a single reproductive bout to ensure that post-copulatory choice can be expressed. As a reproductive strategy, polyandry thereby simultaneously allows females to choose multiple mates and creates the opportunity for further cryptic choice enacted through post-copulatory mechanisms. Polyandry has consequently been widely hypothesised to evolve as an adaptation to allow females to avoid inbreeding and therefore produce higher fitness offspring \cite[][]{Zeh1997, Jennions2000, Tregenza2002, Akcay2007}. However, how pre- and post-copulatory strategies of inbreeding avoidance evolve jointly, and how such evolution is predicted to feed back to affect evolution of polyandry, still remains largely unknown. While pre-copulatory inbreeding avoidance can facilitate evolution of polyandry under some restrictive conditions \cite[][]{Duthie}, the opportunity for post-copulatory choice might facilitate evolution of polyandry under a much wider range of conditions by allowing females to bias fertilisation toward non-kin. 

Severe inbreeding depression might cause selection for alleles underlying active pre- and post-copulatory inbreeding avoidance and associated polyandry, but if evolution of one strategy renders its complement redundant for avoiding inbreeding, then selection might instead be sustained over generations for only one strategy. Empirical studies of multiple species have found evidence for female pre-copulatory inbreeding avoidance \cite[e.g.,][]{Lihoreau2007, Kuriwada2011, Kingma2013, Fischer2015}, and for post-copulatory inbreeding avoidance \cite[e.g.,][]{Simmons2006, Bretman2009}. Further, evidence of both pre-copulatory and post-copulatory inbreeding avoidance has been found for a handful of well-studied species, including Trinidadian guppies \cite[\textit{Poecilia reticulata};][]{Gasparini2011, Daniel2015} and house mice \cite[\textit{Mus domesticus};][]{Potts1991, Firman2015}, while \cite{Liu2014} found evidence for pre-copulatory, but not post-copulatory, inbreeding avoidance in an experimental study of cabbage beetles (\textit{Colaphellus bowringi}). Other studies do not find any evidence for inbreeding avoidance despite strong inbreeding depression, suggesting that evolution of pre- or post-copulatory inbreeding avoidance cannot be presumed \cite[e.g.,][]{Frere2010, Reid2014, Reid2015}. These empirical studies demonstrate that all combinations of adaptive pre-copulatory and post-copulatory inbreeding avoidance are possible, but no conceptual framework exists to predict \textit{a priori} what combinations should be favoured by selection when both pre- and post-copulatory inbreeding avoidance, and polyandry, can evolve. 

Frequencies of alleles underlying active pre- and post-copulatory inbreeding avoidance and polyandry are likely to be affected by direct negative selection on phenotypes (i.e., costs). For example, polyandrous females often pay a cost associated with searching or making themselves available to mates, thereby increasing their own risk of predation \cite[e.g.,][]{Rowe1988, Ronkainen1994, Koga1998}. Females that express pre-copulatory choice (e.g., reluctance to mate with relatives) might pay a cost of increased risk of mating failure, or harm caused by sexual conflict \cite[][]{Rowe1994, Kokko2013}. Females that express post-copulatory choice might pay an energetic cost associated with physiological or biochemical mechanisms needed to store sperm and successfully bias fertilisation \cite[e.g.,][]{Gasparini2011, Tuni2013, Fitzpatrick2014b}. If the relative costs affecting pre- and post-copulatory inbreeding avoidance differ, then alleles underlying the less costly inbreeding strategy might become fixed in the population over generations, while alleles underlying the more costly inbreeding strategy might become extinct if their effects are both costly and made redundant by the less costly strategy.

In addition to the relative importance of costs, evolution of pre- or post-copulatory inbreeding avoidance might be precluded if inbreeding avoidance is already acheived through a complementary strategy, thereby reducing net benefits. Previous models considered the fate of a rare allele underlying inbreeding avoidance in a population initially fixed for random mating \cite[e.g.,][; Duthie and Reid \textit{In press}]{Parker1979, Parker2006, Duthie}. Such models thereby isolate the invasion fitness of a single focal strategy. However, when multiple distinct strategies affect realised inbreeding, it cannot be assumed that both will invade simultaneously, nor that invasion fitness of one inbreeding strategy will be independent of another. For example, if pre-adaptation or rapid selection results in the fixation of alleles underlying adaptive pre-copulatory inbreeding avoidance, then alleles underlying post-copulatory inbreeding avoidance might be precluded from invading a population in which post-copulatory choice is not already expressed because the effect that such invading alleles have on decreasing total inbreeding avoidance could be weak or negligible. 

In a recent model, \cite{Duthie} showed that selection for alleles underlying polyandry and pre-copulatory inbreeding avoidance occurred only under highly constrained conditions in which direct costs of polyandry were low, and when very few males were available during a female's initial mate choice but many males were available during a female's additional mate choice(s), or when polyandry was conditionally expressed based on the degree to which a focal female engaged in inbreeding given her initially chosen mate. In the absence of these constraints, expression of polyandry was expected to increase inbreeding with each additional mate because remaining males were more likely to be relatives once available non-relatives were chosen. However, \cite{Duthie} assumed that polyandrous females had no control over the assignment of paternity to their offspring. If after engaging in polyandry, females could bias paternity through post-copulatory mechanisms, then selection for polyandry to avoid inbreeding might be more widely predicted when inbreeding depression is severe \cite[][]{Zeh1997}, potentially feeding back to drive further evolution of pre- and post-copulatory strategies.

Here we use individual-based modelling and \textit{in silico} experiments to test the hypothesis that, given severe inbreeding depression in offspring viability, evolution of post-copulatory inbreeding avoidance causes selection for polyandry. Additionally, we test the hypothesis that costs associated with polyandry and inbreeding strategies will strongly affect evolution of each such that selection for only the less costly inbreeding avoidance strategy will be sustained over generations. Finally, we test the hypothesis that evolution of one inbreeding strategy will be precluded if it is introduced into a population that is already fixed for a complementary inbreeding strategy that is adaptive. We thereby provide a framework for predicting evolution of pre-copulatory and post-copulatory inbreeding avoidance across reproductive systems with different relative costs and strong inbreeding depression in offspring viability.

\section*{Model}

\begin{table}[!ht]
\captionsetup{margin={-1.75in,0.1in}}
\begin{adjustwidth}{-1.75in}{0in}
%\centering
\caption{\color{Gray}Individual traits (A), model parameter values (B), and model variables (C) for an individual-based model of the evolution of polyandry, pre-copulatory inbreeding strategy, and post-copulatory inbreeding strategy.}
\begin{tabular}{lllll}
\hline
A & Trait & Allele & Genotype & Phenotype &
\hline
  & Tendency for polyandry               &   $P_{a}$  &  $P_{g}$  &  $P_{p}$  &
  & Pre-copulatory inbreeding strategy   &   $M_{a}$  &  $M_{g}$  &  $M_{p}$  &
  & Post-copulatory inbreeding strategy  &   $F_{a}$  &  $F_{g}$  &  $F_{p}$  &
  &                                      &            &           &           &
\hline
B & Description & Parameter & & Default value(s) &
\hline

  & Cost of pre-copulatory inbreeding strategy    & $c_{M}$    & & $0$, $0.02$   &
  & Cost of post-copulatory inbreeding strategy   & $c_{F}$    & & $0$, $0.02$   &
  & Focal female's number of offspring            & $n$        & & $8$           &
  & Log-linear slope of inbreeding depression     & $\beta$    & & $3$           &
  & Adult male immigrants per generation          & $\rho$     & & $5$           & 
  & Female carrying capacity                      & $K_{f}$    & & $100$         &
  & Male carrying capacity                        & $K_{m}$    & & $100$         &
  & Mutation rate of alleles                      & $\mu$      & & $0.001$       &
  & Standard deviation of mutation effect size    & $\mu_{SD}$ & & $\sqrt{1/20}$ &
  &                                               &            & &               &
\hline
C & Description & & & Variable &
\hline
  & Coefficient of kinship                                            & & & $k$              &
  & Focal female's number of mates                                    & & & $N_{males}$      &
  & Female $i$'s perceived absolute mate quality of male $j$          & & & $Q^{m}_{ij}$     &
  & Female $i$'s perceived relative mate quality of male $j$          & & & $q^{m}_{ij}$     &
  & Female $i$'s perceived absolute fertilisation quality of male $j$ & & & $Q^{f}_{ij}$     &
  & Female $i$'s perceived relative fertilisation quality of male $j$ & & & $q^{f}_{ij}$     &
  & Viability of a focal female's offspring                           & & & $\Psi_{\textrm{off}}$     &
\hline
\end{tabular}
\end{adjustwidth}
\end{table}

We model evolution of polyandry and inbreeding avoidance through pre- and post-copulatory mechanisms by tracking interactions among discrete individuals in a small focal population that persists over multiple generations (key individual traits, parameter values, and variables are described in Table 1). Individuals within this population are either female or male, and can be related to each other if they share a common ancestor. Each individual has 10 diploid loci and therefore 20 alleles underlying each trait: tendency to engage in polyandry ($P_{a}$), pre-copulatory inbreeding avoidance ($M_{a}$, i.e., 'mating' alleles), and post-copulatory inbreeding avoidance ($F_{a}$ i.e., `fertilisation' alleles). Alleles combine additively to affect genotypes ($P_{g}$, $M_{g}$, and $F_{g}$) and phenotypes ($P_{p}$, $M_{p}$, and $F_{p}$) for each trait. Over generations, we record mean values of alleles underlying polyandry, pre-copulatory inbreeding strategy, and post-copulatory inbreeding strategy. One generation proceeds with females paying costs, expressing polyandry, mating, and fertilising offspring; offspring alleles then mutate and offspring express inbreeding depression; finally, immigrants arrive in the populatoin and density regulation limits population growth.

\subsection*{Costs}

Values of phenotypic traits for tendency to engage in polyandry ($P_{p}$), pre-copulatory inbreeding strategy ($M_{p}$), and post-copulatory inbreeding strategy ($F_{p}$) each have costs. Each cost independently increases the risk of female pre-mating mortality and, because generations are non-overlapping, total reproductive failure as a linear function of its associated trait. The probability that a focal female dies before mating due to the cost of polyandry ($c_{P}$) is therefore $P_{p} \times c_{P}$. Similarly, the probability of pre-mating female mortality due to the cost of pre-copulatory inbreeding strategy ($c_{M}$) and post-copulatory inbreeding strategy ($c_{F}$) is $|M_{p}| \times c_{M}$ and $|F_{p}| \times c_{F}$, repectively, where $|M_{p}|$ is the absolute value of $M_{p}$ and $|F_{p}|$ is the absolute value of $F_{p}$. Absolute values are used for applying costs to inbreeding strategies because both negative and positive $M_{p}$ and $F_{p}$ bias breeding interactions with respect to kinship, while only positive values of $P_{p}$ affect a female's tendency for polyandry; zero and negative $P_{p}$ values always translate to monandrous phenotypes (see below).

\subsection*{Polyandry}

After costs are paid, each remaining female chooses a number of males to mate with ($N_{males}$) based on her genetically determined tendency to engage in polyandry ($P_{p}$). The exact value of $N_{males}$ is calculated by randomly sampling from a Poisson distribution such that $N_{males} = Poisson(P_{p}) + 1$. This random sampling procedure causes some stochastic variation in the number of realised mates a that female chooses, with a mean $N_{males}$ of $P_{p}+1$. A value of one is added to $P_{p}$ to ensure that all females that have not paid a cost choose at least one mate, as low values of $P_{p}$ might return $Poisson(P_{p})$ values of zero.

The value of $P_{p}$ is affected by the summed values of a focal female's 20 $P_{a}$ alleles. Inheritence of all alleles models a diploid genetic architecture in which 10 $P_{a}$ alleles are randomly sampled from each parent with no physical linkage. Values of $P_{a}$ can take any real number \cite[continuum-of-alleles model;][]{Kimura1965, Lande1976, Reeve2000, Bocedi2014}. The tendency for polyandry genotype $P_{g}$ is therefore defined by the sum of all of a focal female's 20 $P_{a}$ values, which might be negative or positive. However, the tendency for polyandry phenotype $P_{p}$ is restricted to zero or positive values because the number of males that a female chooses to mate with cannot be negative. Hence, negative values of $P_{g}$ are translated to $P_{p}$ values of zero such that $P_{p} = 0$ when $P_{g} < 0$, but when $P_{g} \geq 0$, then $P_{p} = P_{g}$ (i.e., the phenotype value of polyandry equals the genotype value of polyandry if and only if the genotype value is greater than or equal to zero, else the phenotype value of polyandry always equals zero). The resulting $P_{p}$ is therefore modelled as a threshold trait \cite[][]{Lynch1998, Roff1996, Roff1998, Duthie}, in which polyandry is expressed discretely at the threshold value of $P_{g} > 0$, but is determined by the consequence of continuous genetic variation arising from the effects of $P_{a}$ alleles that each contribute additively to polyandry. Negative genotype values of $P_{g}$ therefore all have the same phenotype, resulting in consistently monandrous females. In contrast, positive values of $P_{g}$ and therefore $P_{p}$ can lead to different degrees of realised polyandry in the form of discrete numbers of mates ($N_{males}$).

\subsection*{Mating}

Each male within the population is a potential mate for any female to choose. We therefore assume that there are no opportunity costs associated with male mating such that mating with one female would reduce a male's ability to mate with and sire the offspring of any other female \cite[e.g.,][]{Waser1986}. While all males are therefore available as potential mates, we assume that a female mates with her total allotment of $N_{males}$ without replacement, meaning that $N_{males}$ models the total number of distinct males that a female chooses to mate with rather than the number of times that she mates. In the unlikely event that a female's $N_{males}$ exceeds the total number of males in the population, then she mates with all of the males in the population. Most often, $N_{males}$ will be a small subset of the available males in the population \cite[][]{Duthie}, so the male(s) a focal female chooses will reflect her expression of pre-copulatory inbreeding strategy phenotype ($M_{p}$). Like polyandry allele values, pre-copulatory inbreeding strategy allele values ($M_{a}$) can take any real number. However, unlike polyandry, the sum of all of a female's 20 $M_{a}$ alleles equals both her pre-copulatory inbreeding strategy genotype ($M_{g}$) and her pre-copulatory inbreeding strategy phenotype ($M_{p}$). Negative or positive values of $M_{p}$ cause a female to avoid or prefer mating with kin, respectively. Values of $M_{p}=0$ cause females to mate randomly with respect to kinship.

The probability that a focal female $i$ mates with a male $j$ with which she shares some kinship $k_{ij}$ is calculated by first assigning each male a perceived mate quality $Q^{m}_{ij}$. This quality depends upon whether the female has a strategy of avoiding mating with kin ($M_{p}<0$) or preferring mating with kin ($M_{p}>0$; if $M_{p}=0$, all males are assigned a quality of 1, $Q^{m}_{ij}=1$). If a focal female avoids mating with kin, then $Q^{m}_{ij} = (-M_{p} \times k_{ij} + 1)^{-1}$, meaning that the perceived quality of a male decreases linearly with increasingly positive values of $k_{ij}$ and increasingly negative values of $M_{p}$. If a focal female prefers mating with kin, then $Q^{m}_{ij} = M_{p} \times k_{ij} + 1$, meaning that the perceived quality of a male increases with increasingly positive values of $k_{ij}$ and $M_{p}$. After a focal female assigns all males a perceived quality, the sum of all male perceived qualities is calculated, then each male quality is divided by this sum. This transformation results in each male having a relative perceived mate quality $q^{m}_{ij}$, which is constrained to values between zero and one. The value of $q^{m}_{ij}$ then defines the probability that a focal female chooses the male as a mate; female choice is therefore stochastic, meaning that females do not always choose the male of the highest $q^{m}_{ij}$ (e.g., given $M_{p}$ values of $-2$, $-4$, $-6$, and $-8$, an outbred focal female is $1.5$, $2$, $2.5$, and $3$ times more likely to mate with a non-relative than a full-sibling $k_{ij}=0.25$, respectively). For polyandrous females that choose more than one male mate, mates are chosen iteratively such that $Q^{m}_{ij}$ and $q^{m}_{ij}$ are re-calculated for each additional mate choice, and with $Q^{m}_{ij}$ and therefore $q^{m}_{ij}$ values of already chosen males set to zero to ensure mate sampling without replacement.

\subsection*{Fertilisation}

Fertilisation of a female's $n$ offspring occurs after mating, with each of her $n$ offspring being assigned a sire independently and with replacement from the $N_{males}$ with which the female mated. The probability that a male mate successfully sires an offspring is affected by a female's kinship with the potential sire ($k_{ij}$) and her post-copulatory inbreeding strategy ($F_{p}$). As with polyandry and pre-copulatory inbreeding strategy phenotypes, a female's post-copulatory inbreeding strategy phenotype is determined by the summed effects of 20 post-copulatory inbreeding strategy alleles ($F_{a}$), which can take any real number. The sum of all of a female's $F_{a}$ alleles determines her post-copulatory inbreeding strategy genotype ($F_{g}$), which is identical to her post-copulatory inbreeding strategy phenotype ($F_{p}$). As with pre-copulatory inbreeding strategy phenotype ($M_{p}$), negative or positive values of $F_{p}$ correspond to inbreeding avoidance or preference, respectively, and $F_{p}=0$ causes random fertilisation with respect to kinship. 

The probability that an offspring of a focal female $i$ is fertilised by one of $i$'s mates $j$ is calculated by first assigning a perceived fertilisation quality to each $j$, $Q^{f}_{ij}$. Perceived fertilisation quality $Q^{f}_{ij}$ is calculated in the same way as perceived mate quality $Q^{m}_{ij}$, such that if a focal female has a post-copulatory strategy of avoiding inbreeding ($F_{p}<0$), then the perceived quality of male $j$ is $Q^{f}_{ij} = (-F_{p} \times k_{ij} + 1)^{-1}$. If instead a focal female has a post-copulatory strategy of preferring inbreeding ($F_{p}>0$), then the perceived quality of male $j$ is $Q^{f}_{ij} = F_{p} \times k_{ij} + 1$. The perceived qualities of every male that a focal female has mated with are summed, then a relative quality ($q^{f}_{ij}$) is calculated for each male by dividing each individual male mate's $Q^{f}_{ij}$ by this sum. As with relative perceived mate quality values ($q^{m}_{ij}$), this results in values of perceived fertilisation quality $q^{f}_{ij}$ between zero and one. These $q^{f}_{ij}$ values define the probability that the offspring of a focal female $i$ is fertilised by the male $j$. Females produce $n$ offspring, meaning that $q^{f}_{ij}$ values are sampled $n$ times independently and with replacement for each female to determine the realised distribution of sires.

\subsection*{Mutation}

Alleles of a focal female's offspring mutate with a probability of $\mu=0.001$, with mutation being an independent event that does or does not occur for each locus. When a mutation occurs at a locus, a new value is sampled from a random normal distribution with a mean of zero and a standard devation of mutation effect size $\mu_{SD}$, $\mathcal{N}(0, \mu_{SD})$. This new value is then added to the original value of the allele that mutates \cite[][]{Kimura1965, Lande1976, Bocedi2014, Duthie}. The standard deviation of the mutation effect size ($\mu_{SD}$) is set to $\sqrt{1/20}$, which scales the variance of mutation effect size to the total number of alleles affecting $P_{p}$, $M_{p}$, and $F_{p}$ phenotypes \cite[see][]{Kimura1965, Lande1976, Bocedi2014}.

\subsection*{Inbreeding depression}

The viability of a focal female $i$'s offspring ($\Psi_{\textrm{off}}$) decreases as a function of the coefficient of kinship between $i$ and the sire of her offspring $j$ ($k_{ij}$), and a log-linear slope of inbreeding depression $\beta$. We model selection as having an absolute effect on offspring viability (i.e., hard selection), rather than a relative effect on viability (i.e., soft selection), so that the effect of $\beta$ is consistent over generations and across different parameter combinations. Natural selection on tendency for polyandry ($P_{p}$), pre-copulatory inbreeding strategy ($M_{p}$), and post-copulatory inbreeding strategy ($F_{p}$) can therefore be compared across generations and parameter combinations. If offspring viability was instead relative such that the least inbred offspring had a high viability regardless of the absolute value of $k_{ij}$, then the effect of kinship coefficients and inbreeding depression slopes on offspring viability would depend on the population-wide distribution of $k_{ij}$ between females and males, making model inference more difficult. Therefore, we define offspring viability as,
\begin{equation}
\Psi_{\textrm{off}} = e^{-\beta k_{ij}}
\end{equation}
The slope $\beta$ models the number of haploid lethal equivalents that exist as deleterious recessive alleles in the gametes of $i$ and $j$, and which might combine to become homozygous in offspring and reduce viability. Equation 1 assumes that each deleterious recessive allele affects offspring viability independently, meaning that every deleterious recessive allele added will have a multiplicative affect on decreasing total viability \cite[][]{Morton1956, Mills1994}, causing the log-linear relationship between parent kinship and inbreeding depression. We assume that inbreeding depression always decreases offspring viability, $\beta > 0$ (i.e., no outbreeding depression). Therefore, because $k_{ij}$ is constrained to values between zero and one, the product of $-\beta$ and $k_{ij}$ must be a value less than or equal to zero, $-\beta \times k_{ij} \leq 0$. Values of $\Psi_{\textrm{off}}$ must therefore be between zero (if $-\beta \times k_{ij}$ is very negative) and one (if $-\beta \times k_{ij} = 0$). We can therefore define $\Psi_{\textrm{off}}$ as the probability that an offspring is viable, and sample its realised viability using a Bernoulli trial.

\subsection*{Immigration}

After offspring viability is realised, $\rho$ adult immigrants are added to the focal population. The coefficient of kinship between an immigrant and all other individuals (including native individuals and other immigrants) always equals zero ($k_{ij}=0$). Because kinship coefficient values are calculated directly from the pedigree of the focal population \cite[not from individuals' genomes; see][]{Duthie}, the addition of immigrants introduces non-kin into the population at a constant rate and prevents the mean kinship coefficient of the population from asymptoting to a value of 1 over generations. If the focal population was instead closed ($\rho=0$), then all alleles would quickly be expected to become identical-by-descent among descendants of the ancestral population. To avoid having immigrants affect evolution of tendency for polyandry ($P_{p}$), pre-copulatory inbreeding strategy ($M_{p}$), and post-copulatory inbreeding strategy ($F_{p}$), immigrants are always male and can therefore be chosen based on their kinship values of zero but not actively affect reproductive decisions through the expression of $P_{p}$, $M_{p}$, or $F_{p}$. To further avoid having immigrants affect evolutionary dynamics, allele values underlying ($P_{a}$), pre-copulatory inbreeding strategy ($M_{a}$), and post-copulatory inbreeding strategy ($F_{a}$) are randomly sampled from normal distributions with means and standard deviations of $P_{a}$, $M_{a}$, and $F_{a}$ values equal to the means and standard deviations of those in the focal population at the time of immigration. Overall, this model of immigration effectively assumes that the focal population receives immigrants from other nearby populations that are subject to the same selection on $P_{p}$, $M_{p}$, and $F_{p}$.

\subsection*{Density regulation}

To avoid having unrestricted population growth over generations, we introduce a separate carrying capacity on the total number of females ($K_{f}$) and males ($K_{m}$) in the focal population following immigration \cite[][]{Guillaume2009, Duthie}. If at the end of a generation the number of females or males exceeds their respective carrying capacities, then each sex is reduced back to its $K_{f}$ or $K_{m}$ by randomly removing females and males until each sex is at its carrying capacity. Removal of females and males can be interpreted as dispersal outside the focal population, mortality, or as some combination of dispersal and mortality. The remaining females and males within the population form the next generation of adults.

\subsection*{Model analysis}

To analyse our model, we track mean values of alleles underlying tendency for polyandry ($P_{a}$), pre-copulatory inbreed strategy ($M_{a}$), and post-copulatory inbreeding strategy ($F_{a}$) in a focal population. We present evolutionary dynamics of these mean allele values across different parameter combinations to test our three focal hypotheses. 

To test our hypothesis that selection for polyandry is driven by a female's ability to bias fertilisation toward inbreeding avoidance through post-copulatory mechanisms, we compare simulations in which all three phenotypes including tendency for polyandry ($P_{p}$), pre-copulatory inbreeding strategy ($M_{p}$), and post-copulatory strategy ($F_{p}$) evolve, with an identical set of simulations in which $F_{p}$ cannot evolve given $c_{P} = \{0, 0.0025, 0.005,  0.01\}$. To stop $F_{p}$ from evolving, we sever the connection from $F_{a}$ to $F_{p}$ such that all $F_{g}$ genotypes resulting from $F_{a}$ cause random fertilisation with respect to kinship (i.e, $F_{p}=0$). Hence $F_{a}$ alleles have no phenotypic effect and polyandry cannot cause polyandrous females to benefit through an increased ability to bias fertilisation toward inbreeding avoidance. By comparing mean $P_{a}$ values over generations when post-copulatory inbreeding avoidance also evolves versus when fertilisation is random, we test whether or not evolution of post-copulatory inbreeding strategy increases selection for polyandry.

To test the hypothesis that selection for a costly strategy of inbreeding avoidance will be negligible given a less costly alternative strategy, we quantify change in $M_{a}$ and $F_{a}$ over generations in simulations where the cost of pre-copulatory inbreeding strategy $c_{M}=0$ but the cost of post-copulatory inbreeding strategy $c_{F}=0.02$, and where $c_{F}=0$ but $c_{M}=0.02$. We contrast these simulations with evolution of a costly strategy in the absence of evolution of an alternative strategy (e.g., evolution of pre-copulatory inbreeding strategy when post-copulatory inbreeding strategy phenotype is fixed at zero, $F_{p}=0$). A contrast between the baseline expectation for how costs affect evolution of a focal inbreeding strategy in the absence of an alternative strategy, and evolution of a focal strategy given that an alternative strategy also evolves, tests whether or not a relatively low cost alternative strategy decreases selection for a focal strategy. 

To test the hypothesis that the existence of an adaptive inbreeding strategy will decrease the strength of selection on, and thereby preclude the invasion of, a complementary strategy, we first used exploratory simulations to determine evolution of pre-copulatory inbreeding strategy, and post-copulatory inbreeding strategy and associated polyandry, in isolation. To test whether or not pre-copulatory inbreeding avoidance was predicted to evolve when adaptive polyandry to avoid inbreeding was already established, we initiated pre-copulatory inbreeding strategy allele values at zero ($M_{a}=0$), but fixed post-copulatory inbreeding allele values at $F_{a}=-10$ and polyandry allele values at $P_{a}=1$ (i.e., $F_{p}$ and $P_{p}$ did not evolve). Similarly, to test whether or not post-copulatory inbreeding strategy evolved to avoid inbreeding, we initiated post-copulatory inbreeding strategy allele values and polyandry allele values at zero ($F_{a}=0$; $P_{a}=0$), but fixed pre-copulatory inbreeding strategy allele values at $M_{a}=-10$. For reference, because females have 10 diploid loci, when $M_{a}$ or $F_{a}$ alleles were fixed at $-10$, outbred females were $51$ times less likely to choose a full sibling and $13.5$ times times less likely to choose a first cousin than a non-relative in pre- and post-copulatory choice, respectively.

 {\color{Gray}
\begin{adjustwidth}{-2.2in}{0in}
\mysidecaption{0.68}
{%
   \includegraphics[height=18cm]{figures/poly_costs.pdf}%
}%
{%
   \captionof{figures/poly_costs.pdf}\begin{justify}\vspace{10 mm} \textbf{Figure 1:} Mean allele values underlying tendency for polyandry (red lines and shading), pre-copulatory inbreeding strategy (i.e., blue lines and shading), and post-copulatory inbreeding strategy (i.e., black lines and shading), as calculated across all individuals within a population over 40000 generations given strong inbreeding depression. Solid lines represent grand means over 40 replicate populations, and shading around solid lines show standard errors of allele values around grand means. Negative mean allele values reflect strategies of inbreeding avoidance or tendency for monandry, and positive values reflect strategies of inbreeding preference or tendency for polyandry. Panels in the left-hand column show simulations in which post-copulatory inbreeding strategy is fixed for random fertilisation (indicated by the black horizontal line), while panels in the right-hand column show evolution of both pre- and post-copulatory inbreeding avoidance. Costs of polyandry increase with rows from top to bottom. Dotted horizontal black lines in panels show values of zero on the y-axis.\end{justify}{\t}%
}[t]
\end{adjustwidth}
}

In all simulations, we record mean values of $P_{a}$, $M_{a}$, $F_{a}$ in each generation. Each combination of parameter values simulated is replicated $40$ times, and grand mean values and standard error of means is calculated in each generation across replicate simulations. These analyses allow us to infer how allele values change over generations in response to costs, but also in response to the changing values of other alleles and therefore potentially interacting phenotypes (e.g., selection on post-copulatory inbreeding strategy might be weak if polyandry is rare). For all replicates, we set the maximum number of generations to 40000, which exploratory simulations and previous modelling \cite[][]{Duthie} showed to be a sufficient number of generations for inferring long-term dynamics of mean allele values and therefore selection on phenotypes.  For all replicates, we set $\rho=5$ immigrants, which produced a range of kin and non-kin in each generation allowing females could express inbreeding strategies, and $n=8$ offspring, which was sufficient to keep populations consistently at carrying capacities and avoid population extinction. Values of $K_{f}$ and $K_{m}$ were both set to 100 because previous modelling has shown that populations of this size are small enough that mate encounters between kin occur with sufficient frequency for selection on inbreeding strategy, but populations are also not so small that selection is typically overwhelmed by genetic drift (Duthie and Reid \textit{in press}).


\section*{Results}

\subsection*{Will evolution of post-copulatory inbreeding avoidance strengthen selection for polyandry?}

For low costs of polyandry, $c_{P} < 0.005$, selection caused expected polyandry allele values ($P_{a}$) to be positive only when post-copulatory inbreeding avoidance evolved (Figure 1A-D). For all $c_{P}$, $P_{a}$ values were higher when post-copulatory inbreeding avoidance evolved than when it could not evolve (Figure 1). When post-copulatory inbreeding avoidance could not evolve, $P_{a}$ values were negative even in the absence of a direct cost ($c_{P}=0$; Figure 1A) because females that expressed polyandry were more likely to increase their inbreeding due to sampling males without replacement \cite[][]{Duthie}. Strong post-copulatory inbreeding avoidance consistently evolved in all simulations, even when $P_{a}$ values were expected to be slightly negative (Figure 1F,H), because of the threshold nature of the polyandry phenotype. Figure 2 shows how phenotypic polyandry was common even when mean polyandry allele values within a population were expected to be negative, and how monandry occurred even when mean polyandry allele values were expected to be positive. In all simulations, pre-copulatory inbreeding avoidance also evolved ($M_{a}<0$), meaning that both pre- and post-copulatory inbreeding avoidance was predicted, and evolution of adaptive polyandry occurred given sufficiently low costs of polyandry.

\subsection*{Will selection for a costly inbreeding strategy be negligible given a non-costly alternative strategy?}

For simulations in which either pre- or post-copulatory inbreeding strategy allele values were fixed for random choice with respect to kinship ($M_{a}=0$ or $F_{a}=0$, respectively), the alternative costly inbreeding strategy evolved to inbreeding avoidance ($F_{a}<0$ or $M_{a}<0$; Figure 3). However, when the alternative inbreeding strategy could evolve and was not costly, the focal costly inbreeding strategy initially evolved towards inbreeding avoidance, but then evolved toward random choice with respect to kinship given increasingly negative allele values of the alternative strategy over generations (Figure 3B,D). For example, given fixed $F_{a}=0$, grand mean $M_{a}$ values were $-0.868$ in the last $10000$ generations (Figure 3A), meaning that an outbred focal female was ca $5.34$ times less likely to choose a full-sibling and ca $2.10$ times less likely to choose a first cousin than a non-relative given a strong cost of pre-copulatory inbreeding avoidance, $c_{M}=0.02$. In contrast, grand mean $M_{a}$ allele values were $-0.288$ when post-copulatory inbreeding avoidance also evolved. More strongly, given fixed $M_{a}=0$, grand mean $F_{a}$ values were $-1.092$ in the last $10000$ generations (Figure 3C), meaning that an outbred focal female was ca $6.46$ times less likely to choose a full-sibling and ca $2.37$ times less likely to choose a first cousin than a non-relative given a strong cost of pre-copulatory inbreeding avoidance, $c_{F}=0.02$. However, grand-mean $F_{a}$ values were $-0.070$ when pre-copulatory inbreeding strategy evolved, and polyandry allele values were negative. Overall, selection for a costly inbreeding strategy that was adaptive in isolation was greatly weakened given evolution of a less costly strategy.

{\color{Gray}
\begin{adjustwidth}{-2.2in}{0in}
\mysidecaption{0.68}
{%
   \includegraphics[height=10.25cm]{figures/Pp_vals.pdf}%
}%
{%
   \captionof{figures/poly_costs.pdf}\begin{justify}\vspace{0.25 mm} \textbf{Figure 2:} Relationship between polyandry allele values and phenotypes for an individual-based model in which pre- and post-copulatory inbreeding strategies evolve jointly with polyandry. Two simulations with identical starting conditions (default parameter values with no costs) are shown in panels A,B and C,D. Red lines in (A) and (C) show mean value of alleles underlying polyandry for a single replicate simulation over 40000 generations. Positive allele values contribute to polyandry, while negative values contribute to monandry (dotted line shows zero on the y-axis). In the last generation, mean allele value was below (A) or above (C) zero. Nevertheless, due to among-individual variation and the threshold nature of the polyandry phenotype, polyandry persists in the population in (A) and monandry persists in (C). The histograms in (B) and (D) show females' tendency for polyandry phenotypes in the last generation; white and grey shading indicates monandrous and polyandrous females, respectively.\end{justify}{\t}%
}[t]
\end{adjustwidth}
}

\subsection*{Will selection for a costly inbreeding strategy be negligible given existence of an already adaptive alternative strategy?}

When pre-copulatory inbreeding strategy was initialised for random mating, but polyandry and post-copulatory inbreeding strategy allele values were fixed for adaptive inbreeding avoidance (Figure 4A,B), negative pre-copulatory inbreeding allele values became increasingly more frequent ($M_{a}<0$), causing inbreeding avoidance in mate choice. However, $M_{a}$ values were not as negative after $40000$ generations when post-copulatory inbreeding avoidance and polyandry were fixed as they were when no post-copulatory inbreeding avoidance occurred (e.g., compare Figure 4A with Figure 1A, and Figure 4B with Figure 2A). When post-copulatory inbreeding strategy was initialised for random fertilisation, existence of an already adaptive strategy of pre-copulatory inbreeding avoidance precluded evolution of polyandry and post-copulatory inbreeding avoidance (Figure 4C,D). Mean allele values affecting polyandry consistently decreased over generations regardless of the cost of post-copulatory inbreeding strategy, $c_{F}$. Consequently, when $c_{F}=0$, post-copulatory inbreeding strategy allele values had no effect because females were consistently monandrous, resulting in high drift of $F_{a}$ values among replicates (Figure 4C). However, when $c_{F}=0.02$, $F_{a}$ values remained near zero to minimise direct costs. This lack of evolution of post-copulatory inbreeding avoidance was driven by a lack of polyandry, and therefore an inability of females to bias fertilisation among multiple mates. When pre-copulatory inbreeding avoidance and polyandry were both fixed ($M_{a}=-10$ and $P_{a}=1$), post-copulatory inbreeding strategy evolved to similarly negative mean allele values as pre-copulatory inbreeding strategy did in Figure 4A,B (see Supporting Information). 

{\color{Gray}
\begin{adjustwidth}{-2.2in}{0in}
\mysidecaption{0.31}
{%
   \captionof{figures/poly_costs.pdf}\begin{justify}\vspace{0.25 mm} \textbf{Figure 3:} Mean allele values underlying tendency for polyandry (red lines and shading), pre-copulatory inbreeding strategy (i.e., blue lines and shading), and post-copulatory inbreeding strategy (i.e., black lines and shading), as calculated across all individuals within a population over 40000 generations given strong inbreeding depression. Solid lines represent grand means over 40 replicate populations, and shading around solid lines show standard errors of allele values around grand means. Negative mean allele values reflect strategies of inbreeding avoidance or tendency for monandry, and positive values reflect strategies of inbreeding preference or tendency for polyandry. Panel (A) shows evolution of costly pre-copulatory inbreeding strategy when post-copulatory inbreeding strategy is fixed for random fertilisation (solid black horizontal line), while (B) instead allows post-copulatory inbreeding strategy to evolve under otherwise identical conditions. Similarly, panel (C) shows evolution of costly post-copulatory inbreeding strategy when pre-copulatory inbreeding strategy is fixed for random mating (solid blue horizontal line), while (D) allows pre-copulatory inbreeding strategy to evolve under otherwise identical conditions.\end{justify}{\t}%
}
{%
   \includegraphics[height=14.5cm]{figures/rel_costs.pdf}%
}%
[t]
\end{adjustwidth}
}


\section*{Discussion}

We have provided a novel framework for approaching empirical studies of inbreeding avoidance, predicting evolution of polyandry and inbreeding avoidance through pre- and post-copulatory mechanisms. Accordingly, we modelled evolution of three separate traits affecting polyandry and inbreeding avoidance, including tendency to engage in polyandry, pre-copulatory inbreeding strategy, and post-copulatory inbreeding strategy. Using \textit{in silico} experiments, we tested three specific hypothesis to predict evolution of polyandry, and pre- and post-copulatory inbreeding strategy. We found that (1) selection for polyandry was greatly strengthened given concurrent evolution of post-copulatory inbreeding avoidance, (2) selection for costly pre- or post-copulatory inbreeding avoidance was negligible given evolution of a less costly strategy of inbreeding avoidance, and (3) pre-existence of adaptive pre-copulatory inbreeding avoidance precluded evolution of polyandry and post-copulatory inbreeding avoidance, but not vice versa.

{\color{Gray}
\begin{adjustwidth}{-2.2in}{0in}
\mysidecaption{0.68}
{%
   \includegraphics[height=14cm]{figures/rel_costs.pdf}%
}%
{%
   \captionof{figures/fixed_adapt.pdf}\begin{justify}\vspace{0.25 mm} \textbf{Figure 4:} Mean allele values underlying tendency for polyandry (red lines and shading), pre-copulatory inbreeding strategy (i.e., blue lines and shading), and post-copulatory inbreeding strategy (i.e., black lines and shading), as calculated across all individuals within a population over 40000 generations given strong inbreeding depression. Solid lines represent grand means over 40 replicate populations, and shading around solid lines show standard errors of allele values around grand means. Negative mean allele values reflect strategies of inbreeding avoidance or tendency for monandry, and positive values reflect strategies of inbreeding preference or tendency for polyandry. Panel (A) shows evolution of pre-copulatory inbreeding strategy given fixed polyandry (solid red line) and inbreeding avoidance (solid black line), and (B) shows these same conditions given a cost of pre-copulatory inbreeding strategy. Similarly, panel (C) shows evolution of polyandry and post-copulatory inbreeding strategy when pre-copulatory inbreeding strategy is fixed for inbreeding avoidance (solid blue line), while (D) shows these same conditions given a cost of post-copulatory inbreeding strategy.\end{justify}{\t}%
}[t]
\end{adjustwidth}
}

The opportunity to adjust inbreeding has been widely considered as a driver of adaptive evolution of polyandry \cite[][]{Tregenza2002, Foerster2003, Akcay2007, Varian-Ramos2012, Kingma2013, Lehtonen2015, Reid2014}. Recently, \cite{Duthie} found the conditions under which selection for polyandry and pre-copulatory inbreeding avoidance is predicted to be highly restrictive, requiring strong constraints on the availability of initial compared to additional mates, or the conditional expression of polyandry such that females only engaged in polyandry when their initial choice of mate was of low quality. However, \cite{Duthie} assumed that fertilisation of offspring was unbiased with respect to mate kinship, meaning that females could not express post-copulatory choice. In contrast, we found that when post-copulatory inbreeding avoidance could evolve, selection for polyandry was greatly strengthened (Figure 1). The proposition that polyandry might facilitate cryptic female choice among males of varying compatibility is not new \cite[e.g.,][]{Zeh1997}, but the existence of selection for post-copulatory inbreeding avoidance causing feedback to select for polyandry has not been formally modelled. Our model therefore has widespread implications for empirical studies of evolution of polyandry. 

We predict evolution of polyandry in populations where inbreeding depression is severe and evolution of inbreeding avoidance through post-copulatory mechanisms is not constrained. Post-copulatory mechanisms of inbreeding avoidance have been observed under these conditions in experimental systems across diverse taxa \cite[e.g.,][]{Pizzari2004, Firman2008, Bretman2009, Gasparini2011, Tuni2013, Firman2015}. For example, \cite{Pizzari2004} found that the number of sperm arriving to eggs is lower for full siblings of female red jungefowl (\textit{Gallus gallus}) than it is for non-relatives, even after controlling for effects of order of oviposition and social familiarity. Similarly, Firman and Simmons \citeyearpar{Firman2008, Firman2015} have shown evidence of post-copulatory inbreeding avoidance in house mice, finding fertilisation to be biased toward unrelated males independent of the order in which mating occurs \cite[][]{Firman2008}, perhaps due to post-copulatory mechanisms expressed through the secretion of gametic proteins originating in female's eggs \cite[][]{Firman2015}. Interestingly, female house mice have also been shown to express pre-copulatory inbreeding avoidance  \cite[][]{Potts1991, Roberts2003}. Our results suggest that post-copulatory inbreeding avoidance is unlikely to evolve when females can already avoid inbreeding through pre-copulatory mechanisms (Figure 4), meaning that evolution of post-copulatory inbreeding avoidance could reasonably be hypothesised to have preceded evolution of pre-copulatory inbreeding avoidance in mice.  

Specific mechanisms of post-copulatory inbreeding avoidance are difficult to identify, but recent experimental studies have shown how females can successfully bias fertilisation after mating has occurred in both vertebrates \cite[e.g.,][]{Gasparini2011} and invertebrates \cite[e.g.,][]{Tuni2013}. For example, \cite{Gasparini2011} artificially inseminated female guppies (\textit{Poecilia reticulata}) with equal quantities of sperm from full-siblings and unrelated males, finding that fertilisation success was higher in non-relatives because the velocity of full siblings' sperm was reduced by females' ovarian fluids. Biased fertilisation toward non-relatives might also be accomplished through selective storage of sperm. For example, in black field crickets (\textit{Teleogryllus commodus}), females attempt to remove the spermataphores of unwanted male mates after copulation, while male mates guard females to prevent spermatophore removal \cite[][]{Bussiere2006}. Even if females are unable to remove spermatophores of unwanted mates, \cite{Tuni2013} found that female \textit{T. commodus} are capable of controlling the transfer of sperm to their spermatheca, most likely through physical control of the spermathecal duct, and can reduce sperm storage despite being guarded by a related male. Our results suggest that evolution of such post-copulatory inbreeding avoidance might increase selection for further polyandry, especially if pre-copulatory inbreeding avoidance is costly (Figure 1).

In our model, a cost free inbreeding strategy precluded a more costly complementary inbreeding strategy from persisting in a focal population (Figure 2), meaning that long-term persistence of both pre- and post-copulatory inbreeding avoidance might not be expected in populations given cost asymmetry. Direct costs therefore strongly affected the evolution of the costly phenotype(s), meaning that quantifying costs is necessary for predicting evolution of phenotypes associated with inbreeding strategies. Quantifying direct costs of phenotypes associated with polyandry and mate choice is empirically challenging \cite[][]{Pomiankowski1987, Kokko2003, Reid2015}. \cite{Pomiankowski1987} categorised four types of costs that are relevant to mate choice, which we suggest are also relevant costs for polyandry, and include costs of elevated risks associated with (1) predation or (2) disease transmission, and costs incurred through (3) loss of time or (4) depletion of energy. In our model, we interpreted a cost of polyandy as an elevated risk of predation as might be incurred while searching for or courting mates \cite[e.g.,][]{Rowe1988, Rowe1994}, a cost of pre-copulatory inbreeding strategy as a lost of time \cite[i.e., risk of not finding a mate in time due to choosiness; e.g.,][]{Kokko2013}, and a cost of post-copulatory inbreeding strategy as depletion of energy. While these costs are biologically realistic, different costs will be relevant for different populations and be realised in different ways. For example, increased risk of disease transmission has been observed for highly polyandrous females \cite[][]{Roberts2015a}, but such a cost would more realistically be applied to the number of realised mates a female has rather than her tendency for engaging in polyandry. Future models will therefore need to carefully consider the magnitudes and realisation of direct costs to accurately predict evolution of inbreeding strategies in populations. 

Another type of cost relevant for polyandry and inbreeding strategy is risk of male harm caused by sexual conflict \cite[e.g.,][]{Arnqvist2005a, Parker2006}. We assumed that males were passive in mating encounters with females, but inbreeding theory predicts that males should benefit from a higher tolerance of inbreeding than females, leading to sexual conflict over inbreeding \cite[][]{Parker1979, Parker2006, Kokko2006, Duthie2015a}. Females that express the choice to avoid inbreeding when mating might therefore pay a cost of harm caused by coercion expressed by related males. This cost might itself evolve over time as male harm coevolves with female resistance \cite[][]{Ronn2007, Perry2011, Wang2015a}, and costs might vary with interactions among male non-relatives and relatives because effects of kin selection are predicted to reduce harm caused by related male suitors \cite[][]{Rankin2011a}. To predict evolution of polyandry and inbreeding strategy in populations characterised by active males and sexual conflict, consideration of male inbreeding strategy and coevolution between male harm and female resistance might therefore need to be considered explicitly. 

Our model assumed that allele values underlying polyandry, pre-copulatory inbreeding avoidance, and post-copulatory inbreeding avoidance were all initialised at zero (causing monandry, and random mating and fertilisation with respect to kinship), and phenotypes evolved simultaneously after initialisation. However, these initial conditions are likely to be atypical, as evolution of pre-copulatory and post-copulatory inbreeding avoidance from ancestral random mating and fertilisation is unlikely to occur simultaneously. Rather, initial evolution of one inbreeding strategy (e.g., pre-copulatory inbreeding avoidance) is more likely to occur in the absence of its complementary strategy (e.g., post-copulatory inbreeding avoidance) if the complementary strategy is evolutionarily constrained. Alternatively, initial evolution of one inbreeding strategy might occur in the context of an already established adaptive strategy for avoiding inbreeding. We showed that selection for post-copulatory but not pre-copulatory inbreeding avoidance was strengthened if its complementary strategy was fixed for random choice with respect to kinship, and that selection for post-copulatory but not pre-copulatory inbreeding avoidance was precluded if its complementary strategy was fixed for adaptive inbreeding avoidance.


\subsection*{Acknowledgments}

This work was funded by a European Research Council Starting Grant to JMR. 

\begin{small}
\bibliography{ms_pre_post}
\bibliographystyle{amnatnat}
\end{small}

\end{document}

