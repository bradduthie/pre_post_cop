\documentclass[10pt,letterpaper]{article}
\usepackage[top=0.85in,left=2.75in,footskip=0.75in,marginparwidth=2in]{geometry}
\usepackage{amssymb}
\usepackage{amsmath}
\usepackage{setspace}
\usepackage{natbib}
\usepackage{rotating}
\usepackage{multirow}
\usepackage{datetime}
\setkomafont{\rmfamily\bfseries\boldmath}
\usepackage{wrapfig,floatrow}
\usepackage{float}
\usepackage[font=small,labelfont=bf]{caption}
\usepackage{mathabx}
\usepackage{color}
\usepackage{wasysym}
\usepackage{sidecap}
%\usepackage{caption}
\usepackage{xargs,caption,changepage,ifthen}
\usepackage[demo]{graphicx}

% use Unicode characters - try changing the option if you run into troubles with special characters (e.g. umlauts)
\usepackage[utf8]{inputenc}

% clean citations
%\usepackage{cite}

% hyperref makes references clicky. use \url{www.example.com} or \href{www.example.com}{description} to add a clicky url
\usepackage{nameref,hyperref}
\hypersetup{
     colorlinks   = true,
     citecolor    = blue,
     linkcolor    = black
}

% line numbers
\usepackage[right]{lineno}

% improves typesetting in LaTeX
\usepackage{microtype}
\DisableLigatures[f]{encoding = *, family = * }

% text layout - change as needed
%\raggedright
\setlength{\parindent}{0.5cm}
\textwidth 5.25in 
\textheight 8.75in

% Remove % for double line spacing
%\usepackage{setspace} 
%\doublespacing

% use adjustwidth environment to exceed text width (see examples in text)
\usepackage{changepage}

% adjust caption style
\usepackage[aboveskip=1pt,labelfont=bf,labelsep=period,singlelinecheck=off]{caption}

% remove brackets from references
\makeatletter
\renewcommand{\@biblabel}[1]{\quad#1.}
\makeatother

% headrule, footrule and page numbers
\usepackage{lastpage,fancyhdr,graphicx}
\usepackage{epstopdf}
\pagestyle{myheadings}
\pagestyle{fancy}
\fancyhf{}
\rfoot{\thepage/\pageref{LastPage}}
\renewcommand{\footrule}{\hrule height 2pt \vspace{2mm}}
\fancyheadoffset[L]{2.25in}
\fancyfootoffset[L]{2.25in}

% use \textcolor{color}{text} for colored text (e.g. highlight to-do areas)
\usepackage{color}

% define custom colors (this one is for figure captions)
\definecolor{Gray}{gray}{.25}

% this is required to include graphics
\usepackage{graphicx}

% use if you want to put caption to the side of the figure - see example in text
\usepackage{sidecap}

% use for have text wrap around figures
\usepackage{wrapfig}
\usepackage[pscoord]{eso-pic}
\usepackage[fulladjust]{marginnote}
\reversemarginpar



%----------------------------------------
%COMMAND FOR DOING SIDE CAPTION.
%----------------------------------------
\newcommandx{\mycaptionminipage}[3][3=c,usedefault]{%
    \begin{minipage}[#3]{#1}%
        \ifthenelse{\equal{#3}{b}}{\captionsetup{aboveskip=0pt}}{}
        \ifthenelse{\equal{#3}{t}}{\captionsetup{belowskip=0pt}}{}
        \vspace{0pt}\centering\captionsetup{width=\textwidth} %Temporarily set caption width
        #2%
    \end{minipage}%
}%
\newcommandx{\mysidecaption}[4][4=c,usedefault]{%
    \checkoddpage%
    \ifoddpage%
        %CASE ODD PAGES
        \mycaptionminipage{\dimexpr\linewidth-#1\linewidth-\intextsep\relax}{#3}[#4]%
        \hfill%
        \mycaptionminipage{#1\linewidth}{#2}[#4]%
    \else%
        %CASE EVEN PAGES
        \mycaptionminipage{#1\linewidth}{#2}[#4]%
        \hfill%
        \mycaptionminipage{\dimexpr\linewidth-#1\linewidth-\intextsep\relax}{#3}[#4]%
    \fi%
}%


% document begins here
\begin{document}
\vspace*{0.35in}

% title goes here:
\begin{flushleft}
{\LARGE
\textbf\newline{Evolution of pre-copulatory and post-copulatory inbreeding behaviour and associated polyandry}
}
\newline
% authors go here:
\\
A. Bradley Duthie\textsuperscript{1,*},
Greta Bocedi\textsuperscript{1},
Ryan R. Germain\textsuperscript{1},
Matthew E. Wolak\textsuperscript{1},
Jane M. Reid\textsuperscript{1}
\\
\bigskip
\bf{1} Institute of Biological and Environmental Sciences, School of Biological Sciences, Zoology Building, Tillydrone Avenue, University of Aberdeen, Aberdeen AB24 2TZ, United Kingdom
%\\
%\bf{2} Affiliation B
\\
\bigskip
*  alexander.duthie@stir.ac.uk, brad.duthie@gmail.com

\end{flushleft}

\section*{Abstract}
\marginpar{
\vspace{0.0cm} % adjust vertical position relative to text with \vspace{} - note that you can enter negative numbers to move margin caption up
\color{Gray} % this gives caption a grey color to set it apart from text body
\textbf{Key words:} % note that \ref{fig1} refers to the corresponding wrapfigure
Inbreeding, inbreeding avoidance, inbreeding depression, mate choice, relatedness, reproductive strategy
}
Inbreeding avoidance has been widely hypothesised to evolve through pre-copulatory and post-copulatory mechanisms as a consequence of inbreeding depression in offspring viability. However, no modelling compares evolution of pre-copulatory and post-copulatory inbreeding avoidance jointly, and given biologically relevant costs of inbreeding strategies. Critically, evolution of post-copulatory inbreeding avoidance is restricted to females that mate multiply, potentially leading to complex evolutionary dynamics among inbreeding strategies and polyandry. Accordingly, we use individual-based modelling to track evolution of polyandry and inbreeding strategies under multiple costs and constraints to address three questions: (1) Does evolution of post-copulatory inbreeding avoidance facilitate evolution of costly polyandry? (2) Is evolution of one costly inbreeding strategy inhibited by another non-costly alternative strategy? And (3) is evolution of one inbreeding strategy precluded by existence of another already adaptive strategy? We show that evolution of polyandry is facilitated by post-copulatory inbreeding avoidance. Evolution of relatively low cost inbreeding avoidance strategies was observed, while selection for functionally redundant high cost strategies decreased over generations. Finally, existence of an adaptive inbreeding strategy precluded evolution of post-copulatory, but not pre-copulatory, inbreeding avoidance. Our model thereby introduces a novel framework for predicting evolution of both pre-copulatory and post-copulatory inbreeding strategies in empirical systems.

\section*{Introduction}

Inbreeding often greatly reduces the fitness of inbred offspring \cite[termed `inbreeding depression';][]{Charlesworth1999, Keller2002, Charlesworth2009}. Such strong inbreeding depression is widely hypothesised to drive evolution of inbreeding avoidance, which can be enacted through multiple reproductive decisions in species with obligate biparental reproduction \cite[][]{Parker1979, Parker2006, Pusey1996, Szulkin2012}. From the perspective of a focal breeding female, inbreeding avoidance might be achieved by avoiding mating with related males (i.e., pre-copulatory inbreeding avoidance), or by biasing fertilisation towards unrelated males after mating has occurred, meaning that related males are less likely to sire offspring (i.e., post-copulatory inbreeding avoidance). Importantly, evolution of post-copulatory inbreeding avoidance requires that females are polyandrous, defined as mating with multiple males during a single reproductive bout. Further, such polyandry could itself evolve because it allows females to mate with additional unrelated males following initial mating with a relative, thereby facilitating pre-copulatory inbreeding avoidance even in the absence of any post-copulatory female choice or otherwise biased fertilisation (e.g., given a `fair raffle' among sperm provided by females' mates). As a reproductive strategy, polyandry can thereby simultaneously allow females to mate with less closely related males and create the opportunity for further inbreeding avoidance enacted through active post-copulatory choice. Polyandry has consequently been widely hypothesised to evolve as an adaptation to allow females to avoid inbreeding and thereby increase offspring fitness \cite[][]{Zeh1997, Jennions2000, Tregenza2002, Akcay2007}. However, despite such well-established hypotheses, no modelling investigates conditions under which pre-copulatory or post-copulatory inbreeding avoidance, or both strategies, are likely to evolve, or predicts how such evolution will feed back to affect underlying evolution of polyandry.

Meanwhile, studies of diverse species have found evidence for female inbreeding avoidance enacted through pre-copulatory or post-copulatory routes, or both. For example, pre-copulatory inbreeding avoidance has been documented in weevils, birds, and butterflies \cite[e.g.,][]{Kuriwada2011, Kingma2013, Fischer2015}, and post-copulatory inbreeding avoidance has been documented in crickets \cite[e.g.,][]{Simmons2006, Bretman2009}. Evidence of both pre-copulatory and post-copulatory inbreeding avoidance has accumulated for some well-studied species, including Trinidadian guppies \cite[\textit{Poecilia reticulata};][]{Gasparini2011, Daniel2015} and house mice \cite[\textit{Mus domesticus};][]{Potts1991, Firman2015}. While most studies consider either pre-copulatory or post-copulatory inbreeding avoidance in isolation, a small number of studies consider both. Accordingly, \cite{Liu2014} found evidence of pre-copulatory, but not post-copulatory, inbreeding avoidance in cabbage beetles (\textit{Colaphellus bowringi}). Other studies found no evidence of pre-copulatory or post-copulatory inbreeding avoidance despite strong inbreeding depression \cite[e.g.,][]{Reid2014, Reid2015}. These studies demonstrate that diverse combinations of pre-copulatory and post-copulatory inbreeding avoidance occur in nature. However, there is as yet no modelling framework that predicts what combinations should be favoured by selection when both pre-copulatory and post-copulatory inbreeding avoidance, and polyandry, can evolve. Such a framework would allow prediction of the form of mating system evolution across different systems that experience inbreeding depression and allow existing empirical studies to be compared to model predictions.

\subsection*{Verbal theory}

A recent genetically-explicit individual-based model showed that selection for alleles underlying polyandry and pre-copulatory inbreeding avoidance, and resulting joint evolution of these reproductive strategies, occurred only under highly constrained conditions \cite[][]{Duthie}. Specifically, such evolution occurred when direct costs of polyandry were small, when very few males were available for a female's initial mate choice but many were available for additional mate choice(s), or when polyandry was conditionally expressed when a focal female was related to her initial mate \cite[][]{Duthie}. In the absence of these conditions, increasing degrees of polyandry tended to increase the overall degree of inbreeding that females experienced, and hence increase inbreeding depression in offspring fitness. This occurred because polyandrous females that had evolved to avoid inbreeding had already chosen available unrelated males as their initial mates. Their additional mates, chosen from the remaining male population, were therefore increasingly likely to include relatives. Evolution of polyandry purely to facilitate pre-copulatory inbreeding avoidance was consequently restricted.

However, while \cite{Duthie} considered simultaneous evolution of polyandry and pre-copulatory inbreeding avoidance, they did not also allow simultaneous evolution of post-copulatory inbreeding avoidance, and assumed that polyandrous females had no post-copulatory control over fertilisation. If post-copulatory inbreeding avoidance could also evolve, polyandrous females could further bias paternity towards unrelated males within their set of chosen mates. This process might reduce the cost of polyandry stemming from accumulation of related mates, potentially facilitating evolution of polyandry to avoid inbreeding \cite[][]{Zeh1997}, and driving further consequent evolution of pre-copulatory or post-copulatory mate choice strategies. Therefore, while pre-copulatory inbreeding avoidance can facilitate evolution of polyandry under some restrictive conditions \cite[][]{Duthie}, the opportunity for post-copulatory choice might facilitate evolution of polyandry under a much wider range of conditions by allowing females to bias fertilisation toward non-kin. The long-term evolutionary dynamics of pre-copulatory and post-copulatory inbreeding avoidance then become unclear. Strong inbreeding depression might drive initial evolution of both pre-copulatory and post-copulatory inbreeding avoidance and associated polyandry. However, if evolution of polyandry and post-copulatory inbreeding avoidance renders pre-copulatory inbreeding avoidance redundant, or vice versa, then only one strategy might be maintained in the long-term.

Previous models implicitly or explicitly considered the fate of a rare allele underlying pre-copulatory inbreeding avoidance in a population initially fixed for random mating \cite[e.g.,][]{Parker1979, Parker2006, Duthie, Duthie2016a}. Such models thereby isolate the invasion fitness of this single strategy. However, when both pre-copulatory and post-copulatory strategies can affect realised inbreeding, it cannot be assumed that both strategies will invade simultaneously, nor that invasion fitness of one strategy will be independent of the other. For example, if pre-adaptation or rapid selection results in the fixation of alleles underlying adaptive pre-copulatory inbreeding avoidance, then alleles underlying post-copulatory inbreeding avoidance might be precluded from invading a population in which post-copulatory choice is not already expressed because the effect of such invading alleles could be weak or negligible. 

Ultimately, the absolute and relative frequencies of alleles underlying pre-copulatory and post-copulatory inbreeding avoidance and polyandry are likely to be affected by direct negative selection on resulting phenotypes (i.e., costs) and hence on the net costs versus benefits stemming from inbreeding avoidance. Polyandrous females can pay costs of mate searching or mating, for example, because these activities increase predation risk \cite[e.g.,][]{Rowe1988, Ronkainen1994, Koga1998}. Females that express pre-copulatory choice (e.g., reluctance to mate with relatives) might pay a cost of increased risk of mating failure, or harm caused by sexual conflict \cite[][]{Rowe1994, Kokko2013}. Females that express post-copulatory choice might pay an energetic cost associated with physiological or biochemical mechanisms needed to store sperm and successfully bias fertilisation \cite[e.g.,][]{Gasparini2011, Tuni2013, Fitzpatrick2014b}. If the relative costs affecting pre-copulatory and post-copulatory inbreeding avoidance differ, then alleles underlying the less costly inbreeding strategy might become fixed in the population over generations, while alleles underlying the more costly inbreeding strategy might become extinct if their effects are both costly and made redundant by the less costly strategy.

Here we use individual-based modelling to set up \textit{in silico} experiments, which test the hypothesis that, given strong inbreeding depression in offspring viability, evolution of post-copulatory inbreeding avoidance facilitates evolution of polyandry. Additionally, we test the hypothesis that costs associated with polyandry and inbreeding strategies will strongly affect evolution such that only the less costly inbreeding avoidance strategy will be expressed in the long term. Finally, we test the hypothesis that evolution of one inbreeding avoidance strategy (i.e., pre-copulatory or post-copulatory) will be precluded if it is introduced into a population that is already fixed for the other inbreeding avoidance strategy that is adaptive. We thereby provide a framework for predicting evolution of pre-copulatory and post-copulatory inbreeding avoidance across different reproductive systems.


\section*{Methods}

We model evolution of polyandry and pre-copulatory and post-copulatory inbreeding avoidance by tracking reproductive interactions among individual females and males in a small focal population. Individuals are defined as related if they share a common ancestor within the population pedigree (i.e., non-zero kinship). Each individual has 10 diploid loci, and therefore 20 alleles, underlying each of the three reproductive strategy traits: tendency for polyandry ($P_{a}$), pre-copulatory inbreeding avoidance ($M_{a}$, i.e., `mating' alleles), and post-copulatory inbreeding avoidance ($F_{a}$, i.e., `fertilisation' alleles). Alleles can take any real number \cite[i.e., a continuum-of-alleles model;][]{Kimura1965, Lande1976, Reeve2000, Bocedi2014} and combine additively to affect genotypes ($P_{g}$, $M_{g}$, and $F_{g}$) and phenotypes ($P_{p}$, $M_{p}$, and $F_{p}$) for each trait. In overview, each generation proceeds with females paying costs, expressing polyandry, mating, and fertilisation. Offspring inherit a randomly sampled allele from each parent at each locus with no physical linkage. Alleles can then mutate and offspring express inbreeding depression. Immigrants arrive in the population and density regulation limits population growth. Key individual traits, parameter values, and variables are described in Table 1.

\vspace{5 mm}

\begin{table}[!ht]
\captionsetup{margin={-1.75in,0.1in}}
\begin{adjustwidth}{-1.75in}{0in}
%\centering
\caption{\color{Gray}Individual traits (A), model parameter values (B), and model variables (C) for an individual-based model of the evolution of polyandry, pre-copulatory inbreeding strategy, and post-copulatory inbreeding strategy.}
\begin{tabular}{lllll}
\hline
A & Trait & Allele & Genotype & Phenotype &
\hline
  & Tendency for polyandry               &   $P_{a}$  &  $P_{g}$  &  $P_{p}$  &
  & Pre-copulatory inbreeding strategy   &   $M_{a}$  &  $M_{g}$  &  $M_{p}$  &
  & Post-copulatory inbreeding strategy  &   $F_{a}$  &  $F_{g}$  &  $F_{p}$  &
  &                                      &            &           &           &
\hline
B & Description & Parameter & & Default value(s) &
\hline

  & Cost of pre-copulatory inbreeding strategy    & $c_{M}$    & & $0$, $0.02$   &
  & Cost of post-copulatory inbreeding strategy   & $c_{F}$    & & $0$, $0.02$   &
  & Focal female's number of offspring            & $n$        & & $8$           &
  & Log-linear slope of inbreeding depression     & $\beta$    & & $3$           &
  & Adult male immigrants per generation          & $\rho$     & & $5$           & 
  & Female carrying capacity                      & $K_{f}$    & & $100$         &
  & Male carrying capacity                        & $K_{m}$    & & $100$         &
  & Mutation rate of alleles                      & $\mu$      & & $0.001$       &
  & Standard deviation of mutation effect size    & $\mu_{SD}$ & & $\sqrt{1/20}$ &
  &                                               &            & &               &
\hline
C & Description & & & Variable &
\hline
  & Coefficient of kinship                                            & & & $k$              &
  & Focal female's number of mates                                    & & & $N_{males}$      &
  & Female $i$'s perceived absolute mate quality of male $j$          & & & $Q^{m}_{ij}$     &
  & Female $i$'s perceived relative mate quality of male $j$          & & & $q^{m}_{ij}$     &
  & Female $i$'s perceived absolute fertilisation quality of male $j$ & & & $Q^{f}_{ij}$     &
  & Female $i$'s perceived relative fertilisation quality of male $j$ & & & $q^{f}_{ij}$     &
  & Viability of a focal female's offspring                           & & & $\Psi_{\textrm{off}}$     &
\hline
\end{tabular}
\end{adjustwidth}
\end{table}


\subsection*{Costs}

Phenotypic values for tendency for polyandry ($P_{p}$), pre-copulatory inbreeding strategy ($M_{p}$), and post-copulatory inbreeding strategy ($F_{p}$) each have costs that are assumed to independently increase the probability that a focal female will die before mating. The probabilities of pre-mating mortality due to the costs of polyandry ($c_{P}$), pre-copulatory inbreeding strategy ($c_{M}$), and post-copulatory inbreeding strategy ($c_{F}$) are $P_{p} \times c_{P}$, $|M_{p}| \times c_{M}$, and $|F_{p}| \times c_{F}$, repectively. Here $|M_{p}|$ and $|F_{p}|$ are the absolute values of $M_{p}$ and $F_{p}$, respectively. Absolute values are used for applying costs to inbreeding strategies as appropriate because both negative and positive $M_{p}$ and $F_{p}$ values affect the degree of inbreeding. In contrast, only positive values of $P_{p}$ cause females to be polyandrous (see below). Overall, because generations are non-overlapping, a female's probability of total reproductive failure increases as a linear function with each trait value.

\subsection*{Polyandry}

After costs are paid, each remaining female chooses $N_{males}$ males to mate with, where $N_{males}$ is calculated by sampling from a Poisson distribution such that $N_{males} = Poisson(P_{p}) + 1$. This ensures that all surviving females choose at least one mate, and causes some stochastic variation around the expected mean $N_{males}$ of $P_{p}+1$. A female's value of $P_{p}$ depends on the sum of her 20 $P_{a}$ allele values, which might be negative or positive. However, because females cannot mate with a negative number of males, $P_{p}$ is constrained to be zero or positive such that $P_{p} = 0$ when $P_{g} < 0$, but $P_{p} = P_{g}$ when $P_{g} \geq 0$. Polyandry is therefore modelled as a threshold trait \cite[][]{Lynch1998, Roff1996, Roff1998, Duthie} that is influenced by continuous genetic variation but only expressed at the threshold value of $P_{g} > 0$. Any negative $P_{g}$ value therefore generates phenotypic monandry, while positive $P_{g}$ values can lead to different degrees of realised polyandry (i.e., different numbers of mates, $N_{males}$).

\subsection*{Mating}

All males in the population are assumed to be available for any female to choose. We therefore assume that there is no opportunity cost of male mating such that mating with one female would reduce a male's availability to mate with any other female. Individual females mate with their total allotment of $N_{males}$ without replacement, meaning that $N_{males}$ models a female's total number of different mates rather than solely her total number of matings.

Most often, $N_{males}$ will be smaller than the total number of available males \cite[][]{Duthie}. Here, each female chooses her $N_{males}$ mates based on her pre-copulatory inbreeding strategy phenotype ($M_{p}$), which equals her pre-copulatory inbreeding strategy genotype ($M_{g}$) defined as the sum of the values of her 20 $M_{a}$ alleles. Negative or positive $M_{p}$ values cause a female to avoid or prefer mating with kin, respectively. Values of $M_{p}=0$ cause females to mate randomly with respect to kinship.

The probability that a focal female $i$ mates with a male $j$ with which she shares some kinship $k_{ij}$ is calculated by first assigning each male a perceived mate quality $Q^{m}_{ij}$. If the female has a strategy of pre-copulatory inbreeding avoidance ($M_{p}<0$), then $Q^{m}_{ij} = (-M_{p} \times k_{ij} + 1)^{-1}$, meaning that $Q^{m}_{ij}$ decreases linearly with increasingly positive values of $k_{ij}$ and increasingly negative values of $M_{p}$. If the female has a strategy of pre-copulatory inbreeding preference ($M_{p}>0$), then $Q^{m}_{ij} = M_{p} \times k_{ij} + 1$, meaning that $Q^{m}_{ij}$ increases with increasingly positive $k_{ij}$ and $M_{p}$. If $M_{p}=0$, then all males are assigned a quality of 1

After a focal female assigns all males a $Q^{m}_{ij}$ value, each male's value is divided by the sum of all $Q^{m}_{ij}$ values across all males, thereby assigning each male a relative perceived quality $q^{m}_{ij}$, which is constrained to values between zero and one. The value of $q^{m}_{ij}$ then defines the probability that a focal female mates with the male; mating is therefore stochastic, and females do not always mate with the male of the highest $q^{m}_{ij}$. For polyandrous females that choose multiple mates (i.e., $N_{mates}>1$), mates are chosen iteratively such that $Q^{m}_{ij}$ and $q^{m}_{ij}$ are re-calculated for each additional mate choice, and with $Q^{m}_{ij}$ and therefore $q^{m}_{ij}$ values of already chosen males set to zero to ensure mate sampling without replacement. In the unlikely event that a female's $N_{males}$ exceeds the total number of available males, then she simply mates with all males.

\subsection*{Fertilisation}

Following mating, fertilisation occurs such that each of a female's $n$ offspring is independently assigned a sire (with replacement) from the $N_{males}$ with which the female mated. Sire identity depends on female's kinship with each potential sire ($k_{ij}$) and her post-copulatory inbreeding strategy phenotype ($F_{p}$), which equals her post-copulatory inbreeding strategy genotype ($F_{g}$), defined as the sum of the values of her 20 $F_{a}$ alleles. Negative and positive values of $F_{p}$ correspond to post-copulatory inbreeding avoidance or preference, respectively, and $F_{p}=0$ causes random fertilisation with respect to kinship. 

The probability that an offspring of a focal female $i$ is sired by any one of $i$'s mates $j$ is calculated by assigning a perceived fertilisation quality to each $j$, $Q^{f}_{ij}$. Perceived fertilisation quality $Q^{f}_{ij}$ is calculated in the same way as perceived mate quality $Q^{m}_{ij}$, such that if a focal female has a strategy of post-copulatory inbreeding avoidance ($F_{p}<0$), then the perceived quality of male $j$ is $Q^{f}_{ij} = (-F_{p} \times k_{ij} + 1)^{-1}$. If the focal female has strategy of post-copulatory inbreeding preference ($F_{p}>0$), then the perceived quality of male $j$ is $Q^{f}_{ij} = F_{p} \times k_{ij} + 1$. A relative quality ($q^{f}_{ij}$) is then calculated for each male by dividing his $Q^{f}_{ij}$ by the sum of the $Q^{f}_{ij}$ values across all of a female's mates. These $q^{f}_{ij}$ values, which lie between zero and one, define the probability of paternity.

\subsection*{Mutation}

Offsprings' alleles mutate with independent probabilities $\mu=0.001$. When a mutation occurs, a mutation effect is sampled from a normal distribution with a mean of zero and a standard deviation of $\mu_{SD}$ and added to the original allele value \cite[][]{Kimura1965, Lande1976, Bocedi2014, Duthie}. The value of $\mu_{SD}$ is set to $\sqrt{1/20}$, which scales the variance of mutation effect size to the total number of alleles affecting each phenotype.

\subsection*{Inbreeding depression}

The viability of a focal female $i$'s offspring ($\Psi_{\textrm{off}}$) decreases as a log-linear function of her kinship with the sire of her offspring $j$ ($k_{ij}$) and inbreeding depression slope $\beta$,
\begin{equation}
\Psi_{\textrm{off}} = e^{-\beta k_{ij}}
\end{equation}
Here, $\beta$ models the number of haploid lethal equivalents that exist as deleterious recessive alleles in the gametes of $i$ and $j$, and which might be homozygous in offspring and reduce viability. Equation 1 assumes independent allelic effects, generating multiplicative effects on offspring viability \cite[][]{Morton1956, Mills1994}. We model inbreeding depression as having an absolute rather than relative effect on offspring viability (i.e., hard rather than soft selection), so that the effect of $\beta$ is consistent across generations and different parameter combinations. We also assume that inbreeding always decreases offspring viability, $\beta > 0$ (i.e., no outbreeding depression). Therefore, because $k_{ij}$ is constrained to values between zero and one, $-\beta \times k_{ij} \leq 0$. Values of $\Psi_{\textrm{off}}$ must therefore be between zero (if $-\beta \times k_{ij}$ is very negative) and one (if $-\beta \times k_{ij} = 0$). We therefore define $\Psi_{\textrm{off}}$ as the probability that an offspring is viable, and sample its realised viability (versus mortality) using a Bernoulli trial.

\subsection*{Immigration}

After offspring mortality, $\rho$ adult immigrants are added to the focal population. The kinship between an immigrant and all other individuals always equals zero ($k_{ij}=0$). Immigration therefore prevents the mean kinship within the population from asymptoting to one over generations. To ensure that immigrants do not directly affect genotypic or phenotypic values of tendency for polyandry ($P_{p}$), pre-copulatory inbreeding strategy ($M_{p}$), and post-copulatory inbreeding strategy ($F_{p}$), immigrants are always male. Consequently, they can be chosen as females' mates based on their kinship values of zero but do not actively affect reproductive decisions through the expression of $P_{p}$, $M_{p}$, or $F_{p}$. Further, immigrants $P_{a}$, $M_{a}$, and $F_{a}$ allele values are randomly sampled from normal distributions with means and standard deviations equal to those in the focal population at the time of immigration. We thereby effectively assume that the focal population receives immigrants from other nearby populations that are subject to the same selection on $P_{p}$, $M_{p}$, and $F_{p}$.

\subsection*{Density regulation}

To avoid unrestricted population growth, we set separate carrying capacities for the total numbers of females ($K_{f}$) and males ($K_{m}$) in the focal population following immigration \cite[][]{Guillaume2009, Duthie}. If at the end of a generation the number of females or males exceeds $K_{f}$ or $K_{m}$  respectively, then individuals are randomly removed until each sex is at its carrying capacity. Such removal can be interpreted as some combination of dispersal and mortality. The remaining females and males form the next generation of potentially breeding adults.

\subsection*{Simulation and analysis}

To test the hypothesis that evolution of post-copulatory inbreeding avoidance can facilitate evolution of costly polyandry, we compare simulations in which polyandry and pre-copulatory and post-copulatory inbreeding can all evolve, with otherwise identical simulations in which post-copulatory inbreeding strategy cannot evolve given four different costs of polyandry ($c_{P} = \{0, 0.0025, 0.005,  0.01\}$). To prevent $F_{p}$ from evolving, we sever the connection from $F_{a}$ to $F_{p}$ such that all $F_{g}$ genotypes cause random fertilisation with respect to kinship. Hence $F_{a}$ alleles have no phenotypic effect, meaning that polyandrous females cannot enact post-copulatory inbreeding avoidance.

To test the hypothesis that selection for a costly strategy of inbreeding avoidance will be negligible given a less costly alternative strategy, we quantify change in $M_{a}$ and $F_{a}$ over generations in simulations where the cost of pre-copulatory inbreeding strategy $c_{M}=0$ but the cost of post-copulatory inbreeding strategy $c_{F}=0.02$, and vice versa. We compare these simulations with evolution of a costly strategy in the absence of evolution of an alternative strategy (e.g., evolution of pre-copulatory inbreeding strategy when post-copulatory inbreeding strategy phenotype is fixed at zero, $F_{p}=0$).

To test the hypothesis that the existence of adaptive post-copulatory inbreeding avoidance will preclude the invasion of pre-copulatory inbreeding avoidance, or vice versa, we first used exploratory simulations to determine evolution of pre-copulatory inbreeding strategy, and post-copulatory inbreeding strategy and associated polyandry, in isolation. To test whether pre-copulatory inbreeding avoidance would evolve when adaptive polyandry and post-copulatory inbreeding avoidance was already established, we initiated $M_{a}$ allele values at zero, but fixed $F_{a}$ allele values at $-10$ and $P_{a}$ allele values at $1$ (i.e., meaning that $F_{p}$ and $P_{p}$ did not evolve). Similarly, to test whether post-copulatory inbreeding avoidance evolved when pre-copulatory inbreeding avoidance was already established, we initiated $F_{a}$ and $P_{a}$ allele values at zero but fixed $M_{a}$ allele values at $-10$. Consequently, because females have 10 diploid loci, when $M_{a}$ or $F_{a}$ alleles were fixed at $-10$, outbred females were $51$ times less likely to choose a full brother and $13.5$ times times less likely to choose a first cousin than a non-relative in pre-copulatory and post-copulatory choice, respectively.


In all simulations, we record mean values of $P_{a}$, $M_{a}$, $F_{a}$ in each generation. Each combination of parameter values simulated is replicated $40$ times, and grand mean values and standard errors of means are calculated in each generation across replicates. These analyses allow us to infer how allele values change over generations in response to costs, but also in response to the changing values of other alleles and therefore potentially interacting phenotypes (e.g., selection on post-copulatory inbreeding strategy might be weak if polyandry is rare). For all replicates, we set the maximum number of generations to 40000, which exploratory simulations and previous modelling \cite[][]{Duthie} showed to be a sufficient number of generations for inferring long-term dynamics of mean allele values and therefore selection on phenotypes.  For all replicates, we set $\rho=5$ immigrants, which produced a range of kin and non-kin in each generation allowing females to express inbreeding strategies, and $n=8$ offspring, which was sufficient to keep populations consistently at carrying capacities and avoid population extinction. Values of $K_{f}$ and $K_{m}$ were both set to 100 because previous modelling has shown that populations of this size are small enough that mate encounters between kin occur with sufficient frequency for selection on inbreeding strategy, but populations are also not so small that selection is typically overwhelmed by genetic drift \cite[][]{Duthie2016a}.


\section*{Results}

\subsection*{Does evolution of post-copulatory inbreeding avoidance facilitate evolution of costly polyandry?}

When post-copulatory inbreeding avoidance alleles ($F_{a}$) were fixed to zero and $F_{p}$ could not evolve, $P_{a}$ values were always negative (Figure 1A,C,E,G). This shows that there is selection against unconditional polyandry even given zero direct cost ($c_{P}=0$; Figure 1A). This is because polyandrous females sampled more males from the available population and were consequently likely to mate with some relatives and hence produce some inbred offspring with low viability \cite[][]{Duthie}. 

When post-copulatory inbreeding avoidance was allowed to evolve, mean $P_{a}$ values were substantially higher than in comparable simulations where post-copulatory inbreeding avoidance could not evolve (Figure 1B,D,F,H). Allowing evolution of post-copulatory inbreeding avoidance alongside pre-copulatory inbreeding avoidance therefore facilitated evolution of polyandry. However, $P_{a}$ values were expected to be positive only given a low cost of polyandry ($c_{P} < 0.005$, Figure 1B,D). This implies that, even given strong inbreeding depression in offspring viability, inbreeding avoidance might generally be a weak force driving polyandry evolution.

Strong post-copulatory inbreeding avoidance consistently evolved in all simulations where such evolution was allowed (Figure 1B,D,F,H), even when $P_{a}$ values were expected to be slightly negative (i.e., when there is selection against polyandry, Figure 1F,H). This reflects the threshold nature of phenotypic expression of polygenic polyandry, wherein random sampling of alleles means that polyandry is commonly expressed (i.e., $P_{g}>0$) even when mean $P_{a}$ values are negative (Figure 2). This means that there is commonly opportunity for expression of post-copulatory inbreeding avoidance and associated selection that drives evolution of $F_{a}$ allele values.

Strong pre-copulatory inbreeding avoidance also evolved (i.e., $M_{a}<0$) in all simulations, irrespective of $c_{P}$ and irrespective of whether post-copulatory inbreeding avoidance was allowed to evolve (Figure 1).

 {\color{Gray}
\begin{adjustwidth}{-2.2in}{0in}
\mysidecaption{0.68}
{%
   \includegraphics[height=18cm]{figures/poly_costs.pdf}%
}%
{%
   \captionof{figures/poly_costs.pdf}\begin{justify}\vspace{10 mm} \textbf{Figure 1:} Mean allele values underlying tendency for polyandry (red), pre-copulatory inbreeding strategy (blue), and post-copulatory inbreeding strategy (black) from simulations where post-copulatory inbreeding strategy is (A, C, E, and G) fixed to zero (i.e., random fertilisation) or (B, D, F, H) allowed to evolve freely. Costs of polyandry ($c_{p}$) increase across rows from zero (A and B) to 0.01 (G and H). Mean allele values (solid lines) and associated standard errors (shading) are calculated across all individuals within a population over 40000 generations across 40 replicate populations. Negative mean allele values indicate inbreeding avoidance or tendency for monandry, and positive values indicate inbreeding preference or tendency for polyandry. Dotted lines demarcate mean allele values of zero. \end{justify}{\t}%
}[t]
\end{adjustwidth}
}


{\color{Gray}
\begin{adjustwidth}{-2.2in}{0in}
\mysidecaption{0.68}
{%
   \includegraphics[height=10.25cm]{figures/Pp_vals.pdf}%
}%
{%
   \captionof{figures/poly_costs.pdf}\begin{justify}\vspace{0.25 mm} \textbf{Figure 2:} Relationships between (A and C) polyandry allele values and (B and D) monandry and polyandry phenotypes for simulations with default parameter values and zero costs. Red lines in (A) and (C) show mean polyandry allele values across all individuals in a single simulation over 40000 generations. Positive and negative allele values contribute to polyandry and monandry, respectively. In the final generation, mean allele value was below (A) or above (C) zero (demarcated by the dotted line). Nevertheless, due to the threshold nature of expression of the polygenic polyandry phenotype, polyandry and monandry are expressed in both populations. Histograms in (B) and (D) show females' tendency for polyandry phenotypes in the final generation; white and grey shading indicates monandrous and polyandrous females, respectively.\end{justify}{\t}%
}[t]
\end{adjustwidth}
}


\subsection*{Is evolution of one costly inbreeding strategy inhibited by another non-costly alternative strategy?}

When either pre-copulatory or post-copulatory inbreeding strategy allele values were fixed to zero, the alternative inbreeding strategy evolved to inbreeding avoidance even when costly (Figure 3A,C). However, when one inbreeding strategy could evolve and had zero cost, the other strategy initially evolved towards inbreeding avoidance but then back towards random choice across generations following increasing evolution of inbreeding avoidance (i.e., negative allele values) through the former (zero cost) strategy (Figure 3B,D). For example, given fixed $F_{a}=0$, grand mean $M_{a}$ value in the last $10000$ generations was $-0.868$ (Figure 3A), meaning that a female was ca $5.34$ times less likely to choose a full-sibling and ca $2.10$ times less likely to mate with a first cousin than a non-relative given a high cost of pre-copulatory inbreeding avoidance, $c_{M}=0.02$. However, when $F_{a}$ allele values could also evolve, grand mean $M_{a}$ allele value in the last $10000$ generations was only $-0.288$ (Figure 3B). Further, given fixed $M_{a}=0$, grand mean $F_{a}$ value in the last $10000$ generations was $-1.092$ (Figure 3C), meaning that a female was ca $6.46$ times less likely to choose a full-sibling and ca $2.37$ times less likely to choose a first cousin than a non-relative given a high cost of pre-copulatory inbreeding avoidance, $c_{F}=0.02$. However, when $M_{a}$ values could also evolve and $P_{a}$ values were negative, grand mean $F_{a}$ value in the last $10000$ generations was only $-0.070$ (Figure 3D). Evolution of post-copulatory inbreeding avoidance facilitated evolution of polyandry (Figure 3B,C), while monandry evolved in the absence of post-copulatory inbreeding avoidance (Figure 3A,D; polyandry was cost free in all these simulations). Overall, evolution of a costly inbreeding strategy that readily evolved in isolation was greatly impeded by evolution of a less costly strategy.

{\color{Gray}
\begin{adjustwidth}{-2.2in}{0in}
\mysidecaption{0.31}
{%
   \captionof{figures/poly_costs.pdf}\begin{justify}\vspace{0.25 mm} \textbf{Figure 3:} Mean allele values underlying tendency for polyandry (red), pre-copulatory inbreeding strategy (blue), and post-copulatory inbreeding strategy (black) when (A and B) costly pre-copulatory inbreeding strategy can evolve and post-copulatory inbreeding strategy is (A) fixed for random fertilisation or (B) can also evolve, and when (C and D) costly post-copulatory inbreeding strategy can evolve and pre-copulatory inbreeding strategy is (A) fixed for random mating or (B) can also evolve. Mean allele values (solid lines) and associated standard errors (shading) are calculated across all individuals within a population over 40000 generations across 40 replicate populations. Negative mean allele values reflect strategies of inbreeding avoidance or tendency for monandry, and positive values reflect strategies of inbreeding preference or tendency for polyandry.\end{justify}{\t}%
}
{%
   \includegraphics[height=14cm]{figures/rel_costs.pdf}%
}%
[t]
\end{adjustwidth}
}

\subsection*{Will selection for a costly inbreeding strategy be negligible given existence of an already adaptive alternative strategy?}

When polyandry and post-copulatory inbreeding strategy allele values were fixed for adaptive inbreeding avoidance, pre-copulatory inbreeding avoidance evolved (i.e., $M_{a}$ values became increasingly negative; Figure 4A). Such evolution still occurred, but to a much smaller degree, when pre-copulatory inbreeding avoidance was costly (Figure 4B). However, after $40000$ generations, $M_{a}$ values were less negative when post-copulatory inbreeding avoidance and polyandry were fixed at non-zero values than when they also evolved from initial values of zero (e.g., compare the blue lines in Figures 4A versus Figure 1A).

When pre-copulatory inbreeding strategy values were fixed for adaptive inbreeding avoidance, mean $F_{a}$ allele values did not consistently trend toward negative values (Figure 4C,D). Additionally, mean $P_{a}$ allele values consistently decreased over generations regardless of the cost of post-copulatory inbreeding strategy (Figure 4C,D). Consequently, when $c_{F}=0$, $F_{a}$ allele values had no effect because females were almost exclusively monandrous, resulting in high drift of $F_{a}$ values among replicates (illustrated by wide standard errors, Figure 4C). However, when $c_{F}=0.02$, $F_{a}$ values remained near zero to minimise direct costs. This lack of evolution of post-copulatory inbreeding avoidance was driven by a lack of polyandry, and therefore an inability of females to bias fertilisation among multiple mates. When pre-copulatory inbreeding avoidance and polyandry were both fixed ($M_{a}=-10$ and $P_{a}=1$), $F_{a}$ allele values evolved to similarly negative means as $M_{a}$ allele values in Figure 4A,B (see Supporting Information). 

{\color{Gray}
\begin{adjustwidth}{-2.2in}{0in}
\mysidecaption{0.31}
{%
   \captionof{figures/fixed_adapt.pdf}\begin{justify}\vspace{0.25 mm} \textbf{Figure 4:} Mean allele values underlying tendency for polyandry (red), pre-copulatory inbreeding strategy (blue), and post-copulatory inbreeding strategy (black), given (A and B) fixed polyandry and post-copulatory inbreeding avoidance and (C and D) fixed post-copulatory inbreeding avoidance where the evolving inbreeding strategy is (A and C) cost-free or (B and D) costly. Mean allele values (solid lines) and associated standard errors (shading) are calculated across all individuals within a population over 40000 generations across 40 replicate populations. Negative mean allele values reflect strategies of inbreeding avoidance or tendency for monandry, and positive values reflect strategies of inbreeding preference or tendency for polyandry.\end{justify}{\t}%
}
{%
   \includegraphics[height=14cm]{figures/fixed_adapt.pdf}%
}%
[t]
\end{adjustwidth}
}

\section*{Discussion}

We have used \textit{in silico} experiments to test three key hypotheses concerning evolution of pre-copulatory and post-copulatory inbreeding avoidance and associated polyandry. The modelling that we develop provides a novel framework for approaching empirical studies of inbreeding avoidance, using simulations to address relevant knowledge gaps in understanding evolution of polyandry and inbreeding avoidance through pre-copulatory and post-copulatory mechanisms. Accordingly, we modelled evolution of three separate traits affecting polyandry and inbreeding avoidance, including tendency for polyandry, pre-copulatory inbreeding strategy, and post-copulatory inbreeding strategy. 

\subsection*{Evolution of polyandry is facilitated by post-copulatory inbreeding avoidance}

The opportunity to adjust inbreeding has been widely considered as a driver of adaptive evolution of polyandry \cite[][]{Tregenza2002, Foerster2003, Akcay2007, Varian-Ramos2012, Kingma2013, Lehtonen2015, Reid2014}. Recently, \cite{Duthie} found that the conditions under which selection for polyandry and pre-copulatory inbreeding avoidance is predicted are highly restrictive. However, \cite{Duthie} assumed that fertilisation of offspring was unbiased with respect to mate kinship, meaning that females could not express post-copulatory choice. In contrast, we found that when post-copulatory inbreeding avoidance could evolve, selection for polyandry was greatly strengthened (Figure 1). The proposition that polyandry might facilitate cryptic female choice among males of varying compatibility is not new \cite[e.g.,][]{Zeh1997}, but the existence of selection for post-copulatory inbreeding avoidance causing feedback to select for polyandry has not been formally modelled. Our model therefore has widespread implications for future studies of evolution of polyandry. 

We predict evolution of polyandry in populations where inbreeding depression is severe and evolution of inbreeding avoidance through post-copulatory mechanisms is not constrained. Post-copulatory mechanisms of inbreeding avoidance have been observed under these conditions in experimental systems across diverse taxa \cite[e.g.,][]{Pizzari2004, Firman2008, Bretman2009, Gasparini2011, Tuni2013, Firman2015}. For example, the number of sperm arriving to eggs is lower for full siblings of female red jungefowl (\textit{Gallus gallus}) than it is for non-relatives, even after controlling for effects of order of oviposition and social familiarity \cite[][]{Pizzari2004}. Similarly, in house mice, fertilisation was biased toward unrelated males independent of mating order, perhaps due to post-copulatory mechanisms expressed through the secretion of gametic proteins originating in eggs \cite[][]{Firman2008, Firman2015}. Interestingly, female house mice also show pre-copulatory inbreeding avoidance  \cite[][]{Potts1991, Roberts2003}. Our results suggest that post-copulatory inbreeding avoidance is unlikely to evolve when females can already avoid inbreeding through pre-copulatory mechanisms (Figure 4), meaning that evolution of post-copulatory inbreeding avoidance could reasonably be hypothesised to have preceded evolution of pre-copulatory inbreeding avoidance in mice. 

Recent studies have shown how females can successfully bias fertilisation after mating has occurred. For example, when female guppies (\textit{Poecilia reticulata}) were artificially inseminated with equal quantities of sperm from full-siblings and unrelated males, more eggs were fertilised by unrelated males because the velocities of full siblings' sperm were reduced by females' ovarian fluids \cite[][]{Gasparini2011}. In black field crickets (\textit{Teleogryllus commodus}), females attempt to remove the spermatophores of unwanted males after copulation, and are capable of controlling sperm transfer to spermatheca after copulation occurs \cite[][]{Bussiere2006, Tuni2013}. Our results suggest that evolution of such post-copulatory inbreeding avoidance might increase selection for further polyandry, especially if pre-copulatory inbreeding avoidance is costly (Figure 1).

\subsection*{Evolution of a costly inbreeding strategy is inhibited by a non-costly alternative strategy}

In our model, one cost free inbreeding strategy precluded another more costly inbreeding strategy from persisting in a focal population (Figure 3), meaning that long-term persistence of both pre-copulatory and post-copulatory inbreeding avoidance might not be expected in populations given cost asymmetry. Direct costs therefore strongly affected the evolution of the costly phenotype(s), meaning that quantifying costs is necessary for predicting evolution of phenotypes associated with inbreeding strategies. Quantifying direct costs of phenotypes associated with polyandry and mate choice is empirically challenging \cite[][]{Pomiankowski1987, Kokko2003, Reid2015}. \cite{Pomiankowski1987} categorised four types of costs that are relevant to mate choice, which we suggest are also relevant costs for polyandry, and include costs of elevated risks associated with (1) predation or (2) disease transmission, and costs incurred through (3) loss of time or (4) depletion of energy. In our model, we interpreted a cost of polyandy as an elevated risk of predation as might be incurred while searching for or courting mates \cite[e.g.,][]{Rowe1988, Rowe1994}, a cost of pre-copulatory inbreeding strategy as a lost of time \cite[i.e., risk of not finding a mate in time due to choosiness; e.g.,][]{Kokko2013}, and a cost of post-copulatory inbreeding strategy as depletion of energy. While these costs are biologically realistic, different costs will be relevant for different populations and be realised in different ways. For example, increased risk of disease transmission has been observed for highly polyandrous females \cite[][]{Roberts2015a}, but such a cost would more realistically be applied to the number of realised mates a female has rather than her tendency for engaging in polyandry. Future studies will therefore need to carefully consider the magnitudes and realisation of direct costs to accurately predict evolution of inbreeding strategies in populations. 

Another type of cost relevant for polyandry and inbreeding strategy is risk of male harm caused by sexual conflict \cite[e.g.,][]{Arnqvist2005a, Parker2006}. We assumed that males were passive in mating encounters with females, but inbreeding theory predicts that males should benefit from a higher tolerance of inbreeding than females, leading to sexual conflict over inbreeding \cite[][]{Parker1979, Parker2006, Kokko2006, Duthie2015a}. Females that express the choice to avoid inbreeding when mating might therefore pay a cost of harm caused by coercion expressed by related males. This cost might itself evolve over time as male harm coevolves with female resistance \cite[][]{Ronn2007, Perry2011, Wang2015a}, and costs might vary with interactions among male non-relatives and relatives because effects of kin selection are predicted to reduce harm caused by related male suitors \cite[][]{Rankin2011a}. To predict evolution of polyandry and inbreeding strategy in populations characterised by active males and sexual conflict, consideration of male inbreeding strategy and coevolution between male harm and female resistance might therefore need to be considered explicitly. 

\subsection*{Evolution of post-copulatory inbreeding avoidance is precluded by existing pre-copulatory inbreeding avoidance, but not vice versa}

In our model, existence of adaptive pre-copulatory inbreeding avoidance precluded evolution of polyandry and, in turn, evolution of post-copulatory inbreeding avoidance (Figure 3C,D). However, existence of adaptive post-copulatory inbreeding avoidance did not preclude evolution of pre-copulatory inbreeding avoidance (Figure 3A,B). In wild populations, it is unlikely that pre-copulatory and post-copulatory inbreeding avoidance will evolve simultaneously from an ancestral population in which females mate and assign paternity randomly. Rather, for example, initial evolution of pre-copulatory inbreeding avoidance is much more likely to evolve in the absence of evolving post-copulatory inbreeding avoidance due to constraints, or in the presence of already evolved post-copulatory inbreeding avoidance. When framing hypotheses for existence of post-copulatory inbreeding avoidance and polyandry, it might therefore be necessary to consider whether or not inbreeding avoidance already occurs through pre-copulatory mate choice. Additionally, the opportunity for post-copulatory inbreeding avoidance will also naturally depend on the degree to which females are polyandrous. For species in which pre-copulatory inbreeding avoidance occurs and monandry is common, evolution of post-copulatory inbreeding avoidance is especially unlikely. For example, females of the gregarious cockroach \textit{Blatella germanica} (L.) avoid inbreeding and are typically monandrous \cite[][]{Lihoreau2007}; while occurrence of post-copulatory inbreeding avoidance in \textit{B. germanica} has not, to our knowledge, been tested, we predict that such inbreeding avoidance is highly unlikely. Similarly, females of the solitary parasitoid wasp \textit{Venturia canescens} are monandrous, and appear to avoid mating with siblings \cite[][]{Metzger2010, Metzger2010a}. We hypothesise that post-copulatory inbreeding avoidance in \textit{V. canescens} would be highly unlikely to evolve. 

\subsection*{General predictions}

Some broad predictions are possible given the verbal theory that we have developed and the populations that we have simulated. For example, given that post-copulatory inbreeding avoidance is useless for monandrous females and most effective for polyandrous females, while pre-copulatory inbreeding avoidance is perhaps most effective for monandrous females and less effective for polyandrous females, it would be interesting to test whether or not occurrence of post-copulatory inbreeding avoidance covaries positively, and pre-copulatory inbreeding avoidance covaries negatively, with degree of polyandry. In addition to polyandry making post-copulatory inbreeding avoidance relevant, polyandry might also covary positively with post-copulatory inbreeding avoidance due to the feed back that post-copulatory inbreeding avoidance has on facilitating evolution of polyandry itself. Hence, well-grounded biological hypotheses might be derived for the relationship between female multiple mating and mode of inbreeding avoidance. To test such hypotheses, more work is needed to identify the degree to which females of different species engage in polyandrous and the degree to which females express both pre-copulatory and post-copulatory inbreeding avoidance.


\subsection*{Acknowledgments}

This work was funded by a European Research Council Starting Grant to JMR. 

\begin{small}
\bibliography{ms_pre_post}
\bibliographystyle{amnatnat}
\end{small}

\end{document}

